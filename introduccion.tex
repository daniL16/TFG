\chapter*{Introducción}

\section*{}
\emph{In this work, we consider systems of points on the unit sphere related to problems of approximation and numerical integration. Starting with the construction of the spherical armonic space. Next, we get an expression for critical points of the gradient of a sherical armonic. Finally, we get some results about numeric quadrature.
On the second part, we explain the process followed in the participation in a data mining competition. This process consist in data visualization and preprocessing, choose the right algorithm and analysing the results achieved.}


\bigskip 

Uno de los propósitos de este trabajo es determinar conjuntos de puntos para aproximación, interpolación e integración sobre la esfera y sus propiedades geométricas. Las distribuciones de puntos en la esfera unidad tienen aplicaciones que van desde modelos climáticos globales para la  Tierra, mapeado de los campos estacionales y magnéticos de la Tierra, cristalografía, cúpulas geodésicas, modelado de virus, geometría computacional, etc.
 
\medskip
Para ello, en primer lugar, introduciremos de forma constructiva el espacio de los armónicos esféricos sobre el que trabajaremos. Una vez construido dicho espacio, trataremos el Teorema de adicción y sus consecuencias. Estos ingredientes nos permitirán generar una bases ortonormales del espacio de esféricos armónicos.

Una vez asentadas las bases, estudiaremos el gradiente de dichos polinomios y calcularemos sus puntos críticos. Como caso particular, estudiaremos los puntos críticos de dichos polinomios sobre la esfera de dimensión tres y obtendremos una visualización de los resultados obtenidos.

Finalmente, trataremos varios resultados sobre integración numérica.
\bigskip

En la segunda parte del trabajo, explicaremos el proceso seguido durante la participación en una competición de data mining. Dicho proceso consta de las partes relativas a la comprensión del problema, la visualización de los datos con los que vamos a trabajar, la aplicación de técnicas de preprocesamiento para mejorar el conjunto de datos y la elección del algoritmo a usar para construir un modelo de datos. Finalmente, estudiaremos los resultados obtenidos y realizaremos un análisis sobre los mismos. 

\subsection{Objetivos.}
Los objetivos propuestos al inicio del trabajo fueron los siguientes:
\begin{itemize}
	\item Simulación y visualización de distribuciones de puntos sobre la esfera
	\item Estimación numérica de la calidad de las aproximaciones obtenidas mediante estas distribuciones de puntos. 
	\item Enfrentarse a un problema real de data mining, estudiando las posibles soluciones y los resultados obtenidos.
\end{itemize}

El primer objetivo se ha logrado satisfactoriamente, siendo desarrollado en el Capítulo 2. El segundo objetivo (?). 
Por otro lado, el tercer objetivo se ha cumplido, siendo desarrollado en la segunda parte del trabajo.

\subsection{Organización de la memoria.}
\medskip

El Capítulo 1 contiene todo lo referente a la definición y construcción del espacio de polinomios armónicos esféricos.
En el Capítulo 2 se calculará el gradiente de dichos polinomios y se estudiarán sus puntos críticos. Se estudiará el caso particular de dimensión 3, obteniendo una visualización de los puntos obtenidos sobre la esfera.
Finalmente, en el capítulo 3 se estudiaran diferentes resultados de integración numérica.
En cuanto a la segunda parte del trabajo, en el Capítulo 4 se realiza una introducción al problema a tratar. 
En los Capítulos 5 y 6 se describe el proceso de visualización y posterior preprocesamiento del conjunto de datos.
En el Capítulo 7 se describen los algoritmos usados para resolver el problema.
Finalmente, en el Capítulo 8 se resumen los resultados obtenidos y se describen las conclusiones obtenidas.

\medskip 

Para la redacción de los Capítulos 1 y 3 se ha tomado como referencia \cite{libro_esfarm}, mientras que en el Capítulo 2 se ha tomado como referencia \cite{art_grad}. Para la segunda parte del trabajo se han tomado como referencia los libros \cite{data_mining} y \cite[]{}. Además, para desarrollar el Capítulo 7 se ha tomado como referencia \cite{boosting}, \cite{rusboost} y \cite{cusboost}. 