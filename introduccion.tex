\chapter*{Introducción}

In this work, we consider systems of points on the unit sphere related problems of approximation and numerical integration. Starting with the construction of the armonic spherical space. Following we get an expression for critical points of gradient. Finally, we get some results about numeric quadrature.

On the second part, we explain the process followed in the participation in a data mining competition. 

\bigskip 

El propósito de este trabajo es determinar buenos conjuntos de puntos para aproximación, interpolación e integración sobre la esfera y sus propiedades geométricas. Las distribuciones de puntos en la esfera unidad tienen aplicaciones que van desde modelos climáticos globales para la  Tierra, mapeado de los campos gravitacionales y magnéticos de la Tierra, cristalografía, cúpulas geodésicas, modelado de virus, geometría computacional, etc.
 
\medskip

En primer lugar, introduciremos de forma constructiva los armónicos esféricos. Una vez construido el espacio, trataremos el teorema de adicción y sus consecuencias. 

Una vez asentadas las bases, estudiaremos el gradiente de los dichos polinomios y calcularemos sus puntos críticos.

Finalmente, trataremos varios resultados sobre integración numérica.
\medskip

Los objetivos propuestos al inicio del trabajo fueron los siguientes:
\begin{itemize}
	\item Simulación y visualización de distribuciones de puntos sobre la esfera
	\item Estimación numérica de la calidad de las aproximaciones obtenidas mediante estas distribuciones de puntos. 
\end{itemize}

El primer objetivo se ha logrado satisfactoriamente, siendo desarrollado en el Capítulo 2.

\medskip
El Capítulo 1 contiene todo lo referente a la definición y construcción del espacio de polinomios armónicos esféricos.
En el Capítulo 2 se calculará el gradiente de dichos polinomios y se estudiaran sus puntos críticos. Se estudiará el caso particular de dimensión 3, obteniendo una visualización de los puntos obtenidos sobre la esfera.
Finalmente, en el capítulo 3 se estudiaran diferentes resultados de integración numérica.
Para la redacción de los Capítulos 1 y 3 se ha tomado como referencia \cite{libro_esfarm}, mientras que en el Capítulo 2 se ha tomado como referencia \cite{art_grad}.