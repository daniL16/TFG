
\documentclass[11pt,twoside,a4paper]{book}

\newcommand{\sphere}{\mathds{S}^{d-1}}
\newcommand{\esfera}{\mathds{S}^{2}}
\newcommand{\orto}{\mathds{O}^d}
\newcommand{\R}{\mathds{R}^d}
\newcommand{\spharm}{\mathds{Y}^d_n}
\newcommand{\sphint}{\int_{\sphere}}
\usepackage[spanish,activeacute]{babel} % Para poner acentos y ñ en español
\usepackage[T1]{fontenc}
\usepackage[utf8]{inputenc}
\usepackage{lmodern}
\usepackage{amsmath}
\usepackage{amsthm}
\usepackage{amssymb}
\usepackage{latexsym}
\usepackage{graphicx}
\usepackage{enumerate}   
\usepackage[hidelinks=true]{hyperref}
\usepackage{dsfont}
\usepackage{appendix}
\usepackage{graphicx}
\usepackage{float}
%Numeracion de teoremas, proposiciones, etc.
\numberwithin{equation}{section}
\numberwithin{figure}{section}
\theoremstyle{plain}
\newtheorem{thm}{Teorema}[chapter]
  \theoremstyle{definition}
        \newtheorem{defn}[thm]{Definici\'{o}n}
        \theoremstyle{plain}
  \newtheorem{prop}[thm]{Proposici\'{o}n}
        \theoremstyle{definition}
        \newtheorem{example}[thm]{Ejemplo}
  \theoremstyle{remark}
        \newtheorem{rem}[thm]{Nota}
  \theoremstyle{definition}
        \newtheorem{problem}[thm]{Problema}
  \theoremstyle{lem}
	  \newtheorem{lem}[thm]{Lema}
  \theoremstyle{cor}
  \newtheorem{cor}[thm]{Corolario}

\setlength{\oddsidemargin}{2cm}
\setlength{\evensidemargin}{1cm}

\DeclareGraphicsExtensions{.jpg,.pdf,.mps,.png}

\pagestyle{empty}

\begin{document}


\thispagestyle{empty}

%Portada

\begin{center}

%{\Large UNIVERSIDAD DE GRANADA}\\[8mm]
%{\Large Facultad de Ciencias}\\[8mm]
\includegraphics[width=9cm]{simbolo_color.png}\\[2cm]
{\Large\sffamily Doble Grado en Ingeniería Informática y Matemáticas}\\[3mm]


\vspace{2cm}


\begin{Large}
{\slshape\bfseries  EL T\'ITULO DEL TRABAJO\\[6mm]
FIN DE GRADO}
\end{Large}

\vspace{2cm}

\vfill

\begin{large}
{\bf Trabajo Fin de Grado presentado por \\[3mm]
Daniel López García}
\end{large}


\vspace{1.5cm}

\begin{Large}
Curso 2017/18
\end{Large}
\end{center}




%%%%%%%%%%%%%%%%%%%%%%%%% Página en blanco, revés de la primera página

\quad

\thispagestyle{empty}


\newpage 

%%%%%%%%%%%%%%% Página de portada


\thispagestyle{empty}


\begin{center}

{\Large UNIVERSIDAD DE GRANADA}\\[10mm]
%{\Large Facultad de Ciencias}\\[4mm]
%\includegraphics[width=4cm]{escudo_ugr}\\[4mm]
{\Large\sffamily Doble Grado en Ingeniería Informática y Matemáticas}\\[3mm]


\vspace{3cm}


\begin{Large}
{\slshape\bfseries  EL T\'ITULO DEL TRABAJO\\[6mm]
FIN DE GRADO}
\end{Large}

\vspace{3cm}

\vfill

\begin{large}
{\bf Trabajo Fin de Grado presentado por \\[3mm]
Daniel López García}
\end{large}


\vspace{2cm}

\begin{Large}
Curso 2017/18
\end{Large}
\end{center}

\vfill

\noindent
Tutor: Nombre Apellido1 Apellido2

\noindent
Departamento: Matem\'atica Aplicada

\noindent
\'Area de Conocimiento: Matem\'atica Aplicada

\newpage

%%%%%%%%%%%% Página en blanco, revés

\thispagestyle{empty}

\quad

\newpage

\thispagestyle{empty}

\vspace*{2cm}

\begin{flushright}
\parbox{3.5in}
{\small
(P\'agina de agradecimientos si los hay)


Thank you.}

\end{flushright}

\newpage




\newpage
\phantom{o}
\newpage

\pagenumbering{roman}
\setcounter{page}{3}

\renewcommand{\contentsname}{Índice}
\tableofcontents

\newpage
\phantom{o}
\newpage

\pagestyle{headings}

\setcounter{page}{1}
\pagenumbering{arabic}

%Si quieres poner tus propias marcas en el borde superior de las páginas
%\pagestyle{myheadings}
%\def\leftmark{\textsc{Nombre del autor}}
%\def\rightmark{\textsc{Trabajo Fin de Grado}}

\renewcommand{\appendixname}{Apéndices}
\renewcommand{\appendixtocname}{Apéndices}
\renewcommand{\appendixpagename}{Apéndices}

%Cuerpo del trabajo, se incluye por cap´itulos

%\chapter*{Introducción}

\section*{}
\emph{In this work, we consider systems of points on the unit sphere related problems of approximation and numerical integration. Starting with the construction of the armonic spherical space. Following we get an expression for critical points of gradient. Finally, we get some results about numeric quadrature.
On the second part, we explain the process followed in the participation in a data mining competition. This process consist in data visualization and preprocessing, choose the right algorithm and analysing the results achieved.}


\bigskip 

Uno de los propósitos de este trabajo es determinar buenos conjuntos de puntos para aproximación, interpolación e integración sobre la esfera y sus propiedades geométricas. Las distribuciones de puntos en la esfera unidad tienen aplicaciones que van desde modelos climáticos globales para la  Tierra, mapeado de los campos estacionales y magnéticos de la Tierra, cristalografía, cúpulas geodésicas, modelado de virus, geometría computacional, etc.
 
\medskip
Para ello, en primer lugar, introduciremos de forma constructiva el espacio de los armónicos esféricos sobre el que trabajaremos. Una vez construido dicho espacio, trataremos el teorema de adicción y sus consecuencias. Estos ingredientes nos permitirán generar una bases ortonormales del espacio de esféricos armónicos.

Una vez asentadas las bases, estudiaremos el gradiente de los dichos polinomios y calcularemos sus puntos críticos. Como caso particular, estudiaremos los puntos críticos de dichos polinomios sobre la esfera de dimensión tres y obtendremos una visualización de los resultados obtenidos.

Finalmente, trataremos varios resultados sobre integración numérica.
\bigskip

En la segunda parte del trabajo, explicaremos el proceso seguido durante la participación en una competición de data mining. Dicho proceso consta de las partes relativas a la comprensión del problema, la visualización de los datos con los que vamos a trabajar, la aplicación de técnicas de preprocesamiento para mejorar el conjunto de datos y la elección del algoritmo a usar para construir un modelo de datos. Finalmente, estudiaremos los resultados obtenidos y realizaremos un análisis sobre los mismos. 

\subsection{Objetivos.}
Los objetivos propuestos al inicio del trabajo fueron los siguientes:
\begin{itemize}
	\item Simulación y visualización de distribuciones de puntos sobre la esfera
	\item Estimación numérica de la calidad de las aproximaciones obtenidas mediante estas distribuciones de puntos. 
	\item Enfrentarse a un problema real de data mining, estudiando las posibles soluciones y los resultados obtenidos.
\end{itemize}

El primer objetivo se ha logrado satisfactoriamente, siendo desarrollado en el Capítulo 2. El segundo objetivo (?). 
Por otro lado, el tercer objetivo se ha cumplido, siendo desarrollado en la segunda parte del trabajo.

\subsection{Organización de la memoria.}
\medskip

El Capítulo 1 contiene todo lo referente a la definición y construcción del espacio de polinomios armónicos esféricos.
En el Capítulo 2 se calculará el gradiente de dichos polinomios y se estudiaran sus puntos críticos. Se estudiará el caso particular de dimensión 3, obteniendo una visualización de los puntos obtenidos sobre la esfera.
Finalmente, en el capítulo 3 se estudiaran diferentes resultados de integración numérica.
En cuanto a la segunda parte del trabajo, en el Capítulo 4 se realiza una introducción al problema a tratar. 
En los Capítulos 5 y 6 se describe el proceso de visualización y posterior preprocesamiento del conjunto de datos.
En el Capítulo 7 se describen los algoritmos usados para resolver el problema.
Finalmente, en el Capítulo 8 se resumen los resultados obtenidos y se describen las conclusiones obtenidas.

\medskip 

Para la redacción de los Capítulos 1 y 3 se ha tomado como referencia \cite{libro_esfarm}, mientras que en el Capítulo 2 se ha tomado como referencia \cite{art_grad}. Para la segunda parte del trabajo se han tomado como referencia los libros \cite{data_mining} y \cite[]{}. Además, para desarrollar el Capítulo 7 se ha tomado como referencia \cite{boosting}, \cite{rusboost} y \cite{cusboost}. 

\chapter[Esféricos Armónicos]
        {Esféricos Armónicos}
\section{Preliminares}
\subsection{Notación}
Para empezar fijaremos la notación que seguiremos durante el capítulo. Usaremos $d\in\mathds{N}$ para representar la dimensión de un conjunto; en particular, el conjunto $\mathds{R}^d = \{x=(x_1,...,x_d)^T : x_j\in\mathds{R},1 \le j \le d\}$ es el espacio euclídeo de dimensión d con el producto escalar y la norma
$$
(x,y) = \sum_{j=1}^{d} x_jy_j  \qquad |x|=(x,x)^{1/2}  \qquad x,y\in\mathds{R}^d
$$
En $\mathds{R}^d$ usaremos la base canónica
$$
e_1=(1,0,...,0)^T, ..., e_d=(0,0,...,1)^T
$$
y escribiremos $x = \sum_{j=1}^{d} x_je_j, x\in\mathds{R}^d$.
\medskip

Para indicar la dimensión explícitamente usaremos $x_{(d)}$ en lugar de $x$. En tal caso, $x_{(d)} = x_{(d-1)}+x_de_d$ siendo $x_{(d-1)}=(x_1,...,x_{d-1},0)^T$. También usaremos $x_{(d-1)}$ para referirnos al vector (d-1)-dimensional $(x_1,...,x_{d-1},0)^T$.
\medskip

Trabajaremos sobre la esfera unidad $\sphere = \{\xi\in\R : |\xi| = 1\}$. Por simplicidad, llamaremos esfera a $\sphere$.
\begin{defn} Sean $\xi,\eta\in\mathds{S}^{d-1}$, definimos las siguientes distancias:
	\begin{itemize}
		\item La distancia euclídea $|\xi-\eta| = \sqrt{2(1-\xi\eta)}$
		\item La distancia geodésica $\theta(\xi,\eta)=arccos(\xi.\eta)$
	\end{itemize}	
\end{defn}

\begin{rem}Usando que $\frac{2}{\pi} \le sin t \le t,       t\in[0,\pi/2]$ se deduce la siguiente relación entre ambas distancias:
	$$
	\frac{2}{\pi}\theta(\xi,\theta) \le |\xi - \eta| \le \theta(\xi,\eta)
	$$ 
\end{rem}

Para $x =(x_1,...,x_d)$ definimos $x^\alpha = x_1^{\alpha_1}...x_d^{\alpha_d}$. Análogamente,
para el operador gradiente $\triangledown = (\partial_{x_1},...,\partial_{x_d})^T$ definimos
$$
	\triangledown^\alpha = \frac{\partial^{|\alpha|}}{\partial x_1^{\alpha_1}...\partial x_d^{\alpha_d}}
$$
Y finalmente definimos el operador laplaciano como
$$
	\triangle = \triangledown.\triangledown = \sum_{j=1}^{d}(\frac{\partial}{\partial x_j})^2
$$


\begin{defn} Dado $x\in\mathds{(R)^+}$ definimos la función gamma como
	$$
	\Gamma(x) := \int_{0}^{\infty} t^{x-1}e^{-t}dt		
	$$
\end{defn}
\begin{prop}Se verifican las siguientes formulas:
	$$
	\int_{0}^{\infty}  t{x-1}e^{-at^b}dt = b^{-1}a^{-x/b}\Gamma(x/b)  , x,a,b \in \mathds{R}^+
	$$
	
	$$
	\int_{0}^{1} |ln t|^{x-1}dt = \Gamma(x),   x \in \mathds{R}^+
	$$
	
	$$
	\Gamma(x+1) = x \Gamma(x) ,		x\in \mathds{R}^+
	$$
	
	$$
	\Gamma^{(k)}(x) = \int_{0}^{\infty} (ln t)^k t^{x-1} e^{-t} dt,   k\in\mathds{N}_0,x\in\mathds{R}^+
	$$
\end{prop}
\begin{rem}
$\Gamma(1)=1$ y de la tercera fórmula se deduce que $\Gamma(n+1)=n!, n\in\mathds{N}_0$. Es decir, la función $\Gamma$ extiende el operador factorial de los números naturales a los reales positivos.
\end{rem}
\begin{lem} 
	$$
	\Gamma(\frac{1}{2}) = 	\sqrt{\pi}
	$$
	$$
	\Gamma(n+\frac{1}{2})=\frac{(2n)!}{2^{2n}n!} \sqrt{\pi}
	$$
\end{lem}
\begin{defn}Sea $x\in\mathds{R}$ y $n\in\mathds{N}$,el símbolo de Pochhammer se define como
	$$
	(x)_0 = 1, (x)_n=x(x+1)(x+2)...(x+n-1)
	$$
\end{defn}
\begin{prop} Sea $x\in\mathds{R}^+$ entonces
	$$
	(x)_n = \frac{\Gamma(x+n)}{\Gamma(x)}
	$$
\end{prop}

\section{Esféricos Armónicos a partir de Espacios Primitivos.}
Consideramos $\mathds{O}^d$ el conjunto de matrices ortogonales de orden d. Para cualquier $\eta \in \mathds{O}^d$ vector no nulo, $\mathds{O}^d (\eta)= \{ A \in \mathds{O}^d : A\eta = \eta \} $ es el subconjunto de matrices ortogonales que deja el subespacio $span\{\eta\} = \{\alpha \eta : \alpha \in \mathds{R}\}$ invariante.

\begin{defn}
	Sea $f:\mathds{R}^d \to \mathds{C}$ y A$ \in \mathds{R}^{dxd}$, se define $f_A$ como:
	$$
	f_A(x)=f(Ax)   , \forall x \in \mathds{R}^d
	$$
\end{defn}

Consideremos un subespacio $\mathds{V}$ de funciones definidas de $\mathds{R}^d$ a un subconjunto de $\mathds{R}^d$.
\begin{defn}
	Sea $\mathcal{V}$ un subespacio de funciones definidas de $\mathds{R}^d$ a $A \subseteqq \mathds{R}^d$. Se dice que $\mathcal{V}$ es invariante si para  $f \in \mathcal{V}$ y  $A\in\mathds{O}^d$, entonces  $f_A \in \mathcal{V}$.
	Considerando $\mathcal{V}$ un subespacio invariante de un espacio proveniente de un producto escalar se define:
	\begin{itemize}
		\item $\mathcal{V}$ es reducible si  $\mathcal{V} = \mathcal{V}_1 + \mathcal{V}_2$ con $\mathcal{V}_1 \not= \emptyset$, $\mathcal{V}_2 \not= \emptyset$ verificando $\mathcal{V}_1,\mathcal{V}_2$ irreducibles y $\mathcal{V}_1 \perp \mathcal{V}_2$.
		\item $\mathcal{V}$ es irreducible si no es reducible.
		\item $\mathcal{V}$ es primitivo si es invariante e irreducible.
	\end{itemize}
\end{defn}

\begin{prop}\label{prop:1}
Si $f_A=f$ para cualquier $A\in \mathds{O}^d$ entonces f(x) depende de x por medio de |x|, luego f es constante en una esfera de radio arbitrario.
\end{prop}

\begin{proof} Sean $x,y \in \mathds{R}^d$ con |x| = |y|, podemos encontrar una matriz $A \in \mathds{O}^d$ tal que $Ax = y$. Entonces $f(x)=f_A(x)=f(y)$.
	
\end{proof}

\begin{defn}Dado $f:\mathds{R}^d \to \mathds{C}$ se define $span\{f_A : A \in \orto\}$ como el espacio de las series $\sum c_jf_{A_j}$ convergentes con $A_j \in \mathds{O}^d$,$c_j \in \mathds{C}$ 
\end{defn}
De la definición se deduce que $span\{f_A : A \in \mathds{O}^d\}$ es un subespacio de funciones. Además, si $\mathcal{V}$ es un espacio finito dimensional $\mathcal{V} = span\{f_A\}$
\medskip

Introduciremos los espacios de armónicos esféricos de diferentes órdenes como subespacios primitivos de $C(\mathds{S}^{d-1})$. 
\subsection{Espacios de Polinomios Homogéneos.}
Consideramos $\mathcal{H}^d_n$ el espacio de polinomios homogéneos de grado n en d dimensiones. Las funciones son de la forma:
$$
\sum_{|\alpha|=n}a_\alpha x^\alpha, a_\alpha \in \mathds{C}
$$
\begin{example}
	$$
		\mathds{H}^2_2 = \{ a_1x_1^2 + a_2x_1x_2 + a_3x_2^2\} 
   $$
$$		\mathds{H}^2_3 = \{ a_1x_1^3 + a_2x_2^3 + a_3x_1^2x_2 + a_4x_1x_2^2 \}
	$$
\end{example}
\medskip
A continuación vamos a estudiar la dimensión de  $\mathcal{H}^d_n$, llegando a la conclusión de que es un espacio invariante finito dimensional.
Para determinar $dim \mathcal{H}^d_n$ contamos los monomios de grado n, es decir, $x^\alpha$ con $\alpha_i \ge 0$ y verificando $\alpha_1 + \alpha_2 + ... + \alpha_d = n$. Tomamos un conjunto $S=\{1,2,...,n+d-1\}$. Seleccionamos $d-1$ números de dicho conjunto y los llamamos $\beta_i, 1\leq i \leq d-1$. Definimos  $\beta_0 = 0 $ y $\beta_d = n+d$.

Ahora, tomamos $\alpha_i$ como el número elementos de $S$ entre 2 $\beta_i$ consecutivos, es decir, $ \alpha_i =  \beta_i - \beta_{i-1} - 1,  1\leq i \leq d$. Tenemos que $$
\sum_{i=1}^{d} \alpha_i = \sum_{i=1}^{d} \beta_i - \beta_{i-1} - \sum_{i=1}^{d} 1 = \beta_d - d = n+d-d = n
$$
Por tanto tenemos una biyección entre el conjunto de $\alpha_i$ que suman $n$ y el conjunto de $\beta_i$. Finalmente, contamos las distintas elecciones posibles de los $\beta_i$ y tenemos que
$$
dim \mathds{H}^d_n = \begin{pmatrix}
n+d-1 \\
d-1
\end{pmatrix} = \begin{pmatrix}
n+d-1 \\
n
\end{pmatrix}
$$
%Cada $\mathds{H}_n \in \mathds{H}^d_n$ se puede escribir como:
%$$
%	\mathds{H}_n(x) = \sum_{|\alpha| = n } a_\alpha x^\alpha ,   a_\alpha \in \mathds{C}.
%$$
%Para el polinomio $	\mathds{H}_n(x)$ definimos
%$$
%		\mathds{H}_n(\triangledown) = \sum_{|\alpha| = n } a_\alpha \triangledown^\alpha.
%$$
%Dados 2 polinomios cualesquiera $\mathds{H}_n(x)$,
%$$
%\mathds{H}_{n,1}(x) =  \sum_{|\alpha| = n } a_{\alpha,1} x^\alpha ,		\mathds{H}_{n,2}(x) =  \sum_{|\alpha| = n } a_{\alpha,2} x^\alpha   
%$$
%Se sigue que 
%$$
%%cosas raras
%$$
%Luego $(\mathds{H}_{n,1},\mathds{H}_{n,2})_{\mathds{H}_n^d} := \mathds{H}_{n,1}(\triangledown)\overline{\mathds{H}_{n,2}(x)}$ define un producto escalar en $\mathds{H}_n^d$
\begin{defn}
Una función f es armónica si $\triangle f (x) = 0$. 
\end{defn}
\begin{lem}
	Si $\triangle f = 0$, entonces $\triangle f_A = 0, \quad \forall A \in \mathds{O}^d$
\end{lem}
\begin{proof}
	Sea $ y = Ax$, entonces $\triangledown_x = A \triangledown_y$. Como $ A \in \mathds{O}^d$ se tiene que 
	$$
	\triangle_x = \triangledown_x . \triangledown_x = \triangledown_y . \triangledown_y = \triangle_y
	$$ 
\end{proof}
A continuación, vamos a ver un subespacio de $H^d_n$ importante.
\begin{defn}
Llamamos $\mathds{Y}_n(\mathds{R}^d)$ al espacio de los polinomios homogéneos de grado n en $\mathds{R}^d$ que son armónicos.
\end{defn}
\begin{example}
$\mathds{Y}_n(\mathds{R}^d) = \mathds{H}^d_n$ si n = 0 o n = 1\\
Para d = 1, $\mathds{Y}_n(\mathds{R})=\emptyset$ para $n\ge 2$\\
Para d = 2, $\mathds{Y}_n(\mathds{R}^2)$, los polinomios de la forma $(x_1 + ix_2) ^n$ pertenecen a $\mathds{Y}_n(\mathds{R}^2).$ En particular,  $\mathds{Y}_2(\mathds{R}^2)$ está formado por polinomios de la forma $a(x_1^2-x_2^2)+bx_1x_2, \qquad	a,b\in\mathds{C}$
\end{example} 
Calculamos ahora la dimensión de $\mathds{Y}_n(\mathds{R}^d)$. Llamaremos $N_{n,d}$ a la dimensión de $\mathds{Y}_n(\mathds{R}^d)$.
Sea $H_{n}\in\mathds{H}_n^d$, dicho polinomio puede ser escrito de la forma
$$
H_n(x_1,...,x_d) = \sum_{j=0}^{n}(x_d)^jh_{n-j}(x_1,...x_{d-1}),		h_{n-j}\in\mathds{H}_{n-j}^{d-1}
$$
Aplicamos el operador laplaciano a $H_n$,
$$
\triangle_{(d)}H_n(x_{(d)}) = \sum_{j=0}^{n-2}(x_d)^j[\triangle_{(d-1)}h_{n-j}(x_{(d-1)})+(j+2)(j+1)h_{n-j-2}(x_{(d-1)})]
$$
Luego, si $H_n \in \mathds{Y}_n(\mathds{R}^d) $ entonces $\triangle_{(d)}H_n(x_{(d)}) \equiv 0$ y
$$
h_{n-j-2} = -\frac{1}{(j+2)(j+1)}\triangle_{(d-1)}h_{n-j},		0 \le j \le n-2
$$
En consecuencia un armónico homogéneo está únicamente determinado por $h_n \in \mathds{H}_n^{d-1}$ y $h_{n-1} \in \mathds{H}_{n-1}^{d-2}$. De este modo, obtenemos la siguiente relación
$$
N_{n,d} = dim \mathds{H}_n^{d-1}+  \mathds{H}_{n-1}^{d-1}
$$
Usando la formula obtenida para $dim \mathds{H}_n^d$ se tiene que para $d\ge 2$,
$$
N_{n,d} = \frac{(2n+d-2)(n+d-3)!}{n!(d-2!)},	n\in\mathds{N}
$$
\subsection{Armónicos de Legendre y Polinomios de Legendre}
Ahora, nos centraremos en unos armónicos homogéneos especiales, los armónicos de Legendre de grado n en d dimensiones.
\begin{defn}
	Se define los armónicos de Legendre, $L_{n,d}:\mathds{R^d}\to\mathds{R}$ verificando las siguientes condiciones:
	\begin{itemize}
		\item $L_{n,d} \in \mathds{Y}_n(\mathds{R}^d)$ 
		\item $L_{n,d}(Ax) = L_{n,d}(x) \qquad  \forall A \in \orto(e_d), \forall x \in \mathds{R}^d$ 
		\item $L_{n,d}(e_d) = 1$
	\end{itemize}
\end{defn}
\begin{rem}
La segunda condición implica que $h_{n-j}(A_1x_{d-1})=h_{n-j}(x_{d-1}), \forall A_1\in \mathds{O}^{(d-1)},\quad x_{(d-1)}\in\mathds{R}^{d-1}, \quad 0\le j\le n$
\end{rem}
De la proposición \hyperref[]{\ref*{prop:1} }
se deduce que por ser $h_{n-j}$ polinomio homogéneo,$(n-j)$ es par y 
\begin{equation}
h_{n-j}(x_{(d-1)}) = \left\lbrace
\begin{array}{ll}
c_k|x_{(d-1)}|^{2k} & \textup{si } n-j=2k \\
0 & \textup{si } n-j=2k+1
\end{array}
\right.
\end{equation}
Por tanto,
$$
L_{n,d}(x) = \sum_{k=0}^{[n/2]} c_k|x_{(d-1)}|^{2k}(x_d)^{n-2k}
$$
Determinamos ahora los coeficientes $c_k$
$$
c_k = - \frac{(n-2k+2)(n-2k+1)}{2k(2k+d-3)}c_{k-1}, \qquad 1\le k \le [n/2]
$$
Usando la condición de normalidad se tiene que $c_0 = 1$ y $$
c_k = (-1)^k \frac{n!\Gamma(\frac{d-1}{2})}{4^kk!(n-2k)!\Gamma(k+\frac{d-1}{2})}, \qquad 0\le k \le [n/2]
$$
Finalmente, obtenemos la siguiente expresión 
$$
L_{n,d}(x) = n!\Gamma(\frac{d-1}{2})\sum_{k=0}^{[n/2]}(-1)^k\frac{|x_{(d-1)|^{2k}}(x_d)^{n-2k}}{4^kk!(n-2k)!\Gamma(k+\frac{d-1}{2})}
$$
Usando coordenadas polares $x_{(d)}=r\xi_{(d)},\xi_{(d)} = te_d+\sqrt{1-t^2}\xi_{(d-1)}$, definimos el polinomio de Legendre de grado n en d dimensiones, $P_{n,d}(t) = L_{n,d}(\xi_{(d)})$ como la restricción a la esfera unidad del armónico de Legendre. Por tanto $$
P_{n,d}(t)=n!\Gamma(\frac{d-1}{2})\sum_{k=0}^{[n/2]}(-1)^k\frac{(1-t^2)^{k}t^{n-2k}}{4^kk!(n-2k)!\Gamma(k+\frac{d-1}{2})}
$$
\begin{rem}
$P_{n,d}(1)=1$ y $L_{n,d}(x) = L_{n,d}(r\xi_{(d)}) = r^nP_{n,d}(t)$
\end{rem}
\subsection{Esféricos Armónicos}
\begin{defn}
Se llama espacio de esféricos armónicos de orden n en d dimensiones a	$\mathds{Y}^d_n = \mathds{Y}_n(\mathds{R}^d)_{|\mathds{S}^{d-1}}$ 
\end{defn}
De la definición se deduce que un esférico armónico $\mathds{Y}_n \in \mathds{Y}^d_n$ está asociado a un armónico homogéneo $\mathds{H}_n \in \mathds{Y}^d_n$ de la siguiente forma:
$$
\mathds{H}_n(r\xi) = r^n\mathds{Y}_n(\xi)
$$
En consecuencia, $dim \mathds{Y}^d_n= N_{n,d}$
\begin{thm}\label{thm:1}Sea $\mathds{Y}^d \in \mathds{Y}^d_n$ y $\xi\in\mathds{S}^{d-1}$. Entonces $\mathds{Y}_n$ es invariante respecto a $\mathds{O}^d(\xi)$, si y sólo si, $\mathds{Y}_n(\eta)=\mathds{Y}_n(\xi)\mathds{P}_{n,d}(\xi.\eta)    \forall \eta\in\mathds{S}^{d-1}$
\end{thm}
\begin{proof}
$(\Rightarrow)$ Dado que $\xi$ es un vector unitario podemos encontrar $A_1 \mathds{O}^d$ tal que $\xi = A_1e_d$. Sea $\overset{~}{Y_n(\eta)} = Y_n(A_1\eta),\eta\in\sphere{d-1}$, que es invariante respecto a $\orto(e_d)$. De la definición de armónico de Legendre sabemos que $r^n\overset{~}{Y_n(\eta)} = c_1L_{n_d}(r^n\eta), r\ge0, \eta\in\sphere$ con $c_1$ una constante.
\medskip

Por tanto, $\overset{~}{Y_n(\eta)} = c_1 L_{n,d}(\eta)$ y tomando $\eta = e_d$ tenemos que $c_1 = \overset{~}{Y_n(e_d)}$.

\medskip
Finalmente como 
$$
\overset{~}{Y_n(\eta)} = \overset{~}{Y_n(e_d)}\mathds{P}_{n,d}(\eta.e_d)
$$
se tiene que
$$
\mathds{Y}_n(\eta)=\overset{~}{Y_n(A_1^T\eta)}=Y_n(A_1^T\eta)\mathds{P}_{n,d}(A_1^T\eta.e_d)=Y_n(A_1^T\eta)\mathds{P}_{n,d}(\eta.A_1e_d) = \mathds{Y}_n(\xi)\mathds{P}_{n,d}(\xi.\eta)
$$
$(\Leftarrow)$ Obvio
\end{proof}

\section{Teorema de Adición. Consecuencias.}
\begin{thm}
Sea $\{Y_{n,j}:1\le j \le N_{n,d}\}$ una base ortonormal de $\spharm$, es decir,
$$
\int_{\mathds{S}^{d-1}} Y_{n,j}(\eta)\overline{Y_{n,j}(\eta)} d\mathds{S}^{d-1} = \delta_{j,k},\qquad	1 \le j,k \le N_{n,d}
$$
Entonces, 
$$
\sum_{j=1}^{N_{n,d}}Y_{n,j}(\xi)\overline{Y_{n,j}(\eta)} = \frac{N_{n,d}}{|\mathds{S}^{d-1}|}P_{n,d}(\xi.\eta)		\forall \xi,\eta \in \mathds{S}^{d-1}
$$

\end{thm}
\begin{proof}Sean $A\in\orto$ y $1\le k \le N_{n,d}$, $Y_{n,k}(A\xi)\in \spharm$ podemos escribir
$$
Y_{n,k}(A\xi) = \sum_{j=1}^{N_{n,d}} c_{kj}Y_{n,j}(\xi), \qquad  c_{kj}\in\mathds{C}
$$
Como $$
\int_{\sphere} Y_{n,k}(A\xi) \overline{Y_{n,k}(A\xi) } d\sphere(\xi) = \int_{\sphere} Y_{n,k}(\eta) \overline{Y_{n,k}(A\eta) } d\sphere(\eta) =\delta_{j,k}
$$
tenemos que 
$$
\delta_{jk} = \sum_{l,m = 1}^{N_{n,d}} c_{j,l}\overline{c_{k,m}}(Y_{n,l},Y_{n,m}) = \sum_{l,m = 1}^{N_{n,d}} c_{j,l}\overline{c_{k,l}}
$$
Sea $C=(c_{j,l})$ y $C^H$ su matriz conjugada transpuesta. Se verifica que $CC^H=I$ y $C^HC=I$ luego C es unitaria y
$$
\sum_{j=1}^{N_{n,d}} \overline{c_{jl}}c_{jk} = \delta_{lk} \qquad 1\le l,k \le N_{n,d}
$$
Ahora, consideramos la suma
$$
Y(\xi,\eta) = \sum_{j=1}^{N_{n,d}} Y_{n,j}(\xi)\overline{ Y_{n,j}(\eta)} ,\quad \xi,\eta\in\sphere
$$
Para $A\in\orto$ y fijado $\xi$ se tiene que
$$
Y(A\xi,A\eta) = \sum_{j=1}^{N_{n,d}} Y_{n,j}(A\xi)\overline{ Y_{n,j}(A\eta)} = \sum_{j,k,l=1}^{N_{n,d}} c_{jk}\overline{c_{jl}} Y_{n,k}(\xi)\overline{ Y_{n,l}(\eta)} = \sum_{j=1}^{N_{n,d}} Y_{n,k}(\xi)\overline{ Y_{n,k}(\eta)} = Y(\xi,\eta)
$$
luego $Y(\xi,.)\in\spharm$ es invariante respecto a $\orto(\xi)$. Por el teorema \ref{thm:1} $Y(\xi,\eta)=Y(\xi,\xi)P_{n,d}(\xi.\eta)$. Análogamente, $Y(\xi,\eta)=Y(\eta,\eta)P_{n,d}(\xi.\eta)$. En consecuencia, $Y(\xi,\xi) = Y(\eta,\eta)$ y es una constante en $\sphere$. Para determinar dicha constante, integramos la igualdad  $Y(\xi,\xi) =  \sum_{j=1}^{N_{n,d}} |Y_{n,j}(\xi)|^2 $ sobre la esfera, obteniendo que
$$
Y(\xi,\xi)|\sphere| =  \sum_{j=1}^{N_{n,d}}\int_{\sphere} |Y_{n,j}(\xi)|^2 d\sphere = N_{n,d}
$$
Por tanto, $Y(\xi,\xi)=\frac{N_{n,d}}{|\sphere|}$ y se cumple $\sum_{j=1}^{N_{n,d}}Y_{n,j}(\xi)\overline{Y_{n,j}(\eta)}= Y(\xi,\eta)=Y(\xi,\xi)P_{n,d}(\xi.\eta) = \frac{N_{n,d}}{|\mathds{S}^{d-1}|}P_{n,d}(\xi.\eta)	$
\end{proof}
\medskip 

Veamos ahora algunas aplicaciones del teorema de adición. En primer lugar, aplicaremos el teorema para encontrar una expresión reducida del kernel de $\spharm$.
\medskip

Cada $Y_n \in \spharm$ puede escribirse de la forma
$$Y_n(\xi) = \sum_{j=1}^{N_{n,d}} (Y_n,Y_{n,j})_{\sphere} Y_{n,j}(\xi)$$
Aplicando el teorema,
$$Y_(\xi)=\int_{\sphere} Y_n(\eta)\sum_{j=1}^{N_{n,d}}Y_{n,j}(\xi)\overline{Y_{n,j}(\eta)}d\sphere(\eta) = \frac{N_{n,d}}{|\sphere|}\int_{\sphere}P_{n,d}(\xi.\eta)Y_n(\eta)d\sphere(\eta)$$
Por tanto, $$
K_{n,d}(\xi.\eta)P_{n,d}(\xi.\eta)
$$
es el kernel "reproducing" de $\spharm$, es decir, $$
Y_n(\xi) = (Y_n,K_{n,d}(\xi,\cdot))_{\sphere} \qquad \forall Y_n \in \spharm, \xi \in \sphere$$
Definimos $\mathds{Y}_{0:m}^d = \underset{n=0}{\overset{m}{\oplus}} \spharm$ como el espacio de todos los esfericos armónicos de orden menor o igual a m. Entonces 
$${K}_{0:m,d}(\xi,\eta) = \frac{1}{|\sphere|}\sum_{n=0}^{m}N_{n,d}P_{n,d}(\xi.\eta)$$ es el "reproducing kernel" de  $\mathds{Y}_{0:m}^d$.
\medskip

A continuación, obtendremos límites para los esféricos armónicos y los polinomios de Legendre.
\begin{prop}
	\begin{gather}
		||Y_n||_\infty \le \left(\frac{N_{n,d}}{|\sphere|}\right)^{\frac{1}{2}}||Y_n||_{L^2(\sphere)} \\
		|P_{n,d}(t)| \le 1 = P_{n,d}(1)
	\end{gather}
\end{prop}
\begin{proof}Tomando $\xi \in \sphere$ por el teorema de adición $$
	\sum_{j=1}^{N_{n,d}} |Y_{n,j}(\xi)|^2 = \frac{N_{n,d}}{|\sphere|} P_{n,d}(||\xi||^2) = \frac{N_{n,d}}{|\sphere|}$$
\end{proof}
\begin{prop}
	$$\sphint |P_{n,d}(\xi.\eta)|^2 dS^{d-1}(\eta) = \frac{|\sphere|}{N_{n,d}}
	$$
\end{prop}
\begin{proof}

	\begin{gather*}
		\begin{aligned}
	\sphint |P_{n,d}(\xi.\eta)|^2 dS^{d-1}(\eta) &= \\ (\frac{|\sphere|}{N_{n,d}})^2 \sphint |\sum_{j=1}^{N_{n,d}} Y_{n,j}(\xi)\overline{Y_{n,j}(\eta)}|^2 dS^{d-1}(\eta)  &=\\  (\frac{|\sphere|}{N_{n,d}})^2 \sum_{j=1}^{N_{n,d}} |Y_{n,j}(\xi)|^2 &= \frac{|\sphere|}{N_{n,d}}
	\end{aligned}
	\end{gather*}	
	
	
	
\end{proof}
\begin{thm}Para cualquier $n\in\mathds{N}_0$ y $d\in\mathds{N}$ el espacio $\spharm$ es irreducible
\end{thm}
\begin{proof}
Razonamos por deducción al absurdo. Supongamos que $\spharm$ es reducible entonces $\exists V_1,V_2$ no vacíos, verificando que $\spharm = V_1 + V_2$ y $V_1 \perp V_2$. Tomamos una base ortonormal de $\spharm$ tal que las primeras $N_1$ funciones recubren $V_1$ y las restantes $N_2 = N_{n,d} - N_1$ recubren  $V_2$. Podemos aplicar el teorema de adición a $V_1$ y $V_2$ con las funciones de Legendre $P_{n,d,1}$ y $P_{n,d,2}$.
\medskip

Como $V_1 \perp V_2$ 
\begin{equation}\label{eq:1}
\int_{\sphere} P_{n,d,1}(\xi\eta)P_{n,d,2}(\xi\eta)d\sphere(\eta) = 0 \qquad \forall\xi\in\sphere
\end{equation}

Fijamos $\xi\in\sphere$ y sea $\phi$ una función tal que $\phi(\eta) = P_{n,d,1}(\xi.\eta)$. Tomamos $A\in\orto(\xi)$ y se cumple que $A^T\xi= \xi$. Entonces
$$P_{n,d,1}(\xi.A.\eta) = P_{n,d,1}(A^T\xi.\eta)=P_{n,d,1}(\xi.\eta)
$$
es decir, $\phi$ es invariante respecto a $\orto(\xi)$. Por el teorema \hyperref[]{\ref{thm:1}}
$$
P_{n,d,1}(\xi.\eta) = P_{n,d,1}(\xi.\xi).P_{n,d}(\xi.\eta) = P_{n,d}(\xi.\eta)
$$
Razonando de forma análoga para $P_{n,d,2}$ se tiene que
$$
P_{n,d,2}(\xi.\eta) = P_{n,d}(\xi.\eta)
$$
Sin embargo, tenemos que 
$$ 0=\int_{\sphere}P_{n,d,1}(\xi\eta)P_{n,d,2}(\xi\eta)d\sphere(\eta) =\int_{\sphere} |P_{n,d}(\xi\eta)|^2 d\sphere(\eta) = \frac{|\sphere|}{N_{n,d}}
$$
Hemos llegado a una contradicción, por tanto, $\spharm$ es irreducible
\end{proof}
\section{Un Operador de Proyección}
Buscamos la mejor aproximación de una función $f\in L^2(\sphere)$ en $\spharm$, es decir, $inf\{||f-Y_n||_{L^2(\sphere)}: Y_n \in \spharm\}$. Si $\{Y_{n,j} : 1\le j \le N_{n,d}\}$ es una base ortonormal de $\spharm$ entonces la solución es la proyección de f en $\spharm$ que está definido para $f\in L^1(\sphere)$ 
$$ (P_{n,d}f)(\xi) = \sum_{j=1}^{N_{n,d}}(f,Y_{n,f})_{\sphere} Y_{n,j}(\xi)$$


\begin{defn}Se define la proyección de $f\in L^1(\sphere)$ en $\spharm$ como $$
		(P_{n,d}f)(\xi) =\frac{N_{n,d}}{|\sphere|} \int_{\mathds{S}^{d-1}}P_{n,d}(\xi.\eta)f(\eta)d\mathds{S}^{d-1},\qquad \xi\in\sphere
	$$
\end{defn}
\begin{rem}El operador $P_{n,d}$ es lineal
\end{rem}
%mas cosas
\begin{prop}El operador proyección $P_{n,d}$ conmuta con las transformaciones ortogonales, es decir, $P_{n,d}f_A=(P_{n,d}f)_A \quad \forall A\in\orto$
\end{prop}
\begin{proof}
	\begin{gather*}
	\begin{aligned}
	(P_{n,d}f_A) (\xi) &= \frac{N_{n,d}}{|\sphere|} \int_{\sphere}P_{n,d}(\xi.\eta)f(A\eta)d\sphere(\eta) \\&= \frac{N_{n,d}}{|\sphere|} \int_{\sphere}P_{n,d}(A\xi.\zeta)f(\zeta)d\sphere(\zeta) =  
	(P_{n,d}f)_A (\xi)
	\end{aligned}
	\end{gather*}
	
\end{proof}
\begin{cor}Si $\mathds{V}$ es un espacio invariante, entonces $P_{n,d}\mathds{V} = {P_{n,d}f : f\in\mathds{V}}$ es un subespacio invariante de $\spharm$.
\end{cor}
\begin{thm}Si $\mathds{V}$ es un espacio invariante de $C(\sphere)$ entonces o $\mathds{V} \perp \spharm$ o $P_{n,d}$ es una biyección de $\mathds{V}$ sobre $\spharm$. En el último caso, $\mathds{V}=\spharm$
\end{thm}
\begin{proof}
Veamos que si $P_{n,d}:\mathds{V} \to \spharm$ es una biyección entonces $\mathds{V}=\spharm$. Ambos espacios son de dimensión finita y tienen la misma dimensión, $N_{n,d}=dim(\spharm)$. Sea $\{V_j : 1 \le j \le N_{n,d} \}$ una base ortonormal de  $\mathds{V}$. Por ser $\mathds{V}$ primitivo, para cada $A\in\orto$ $$
V_j(A\xi) = \sum_{k=1}^{N_{n,d}}c_{jk}V_k(\xi), \quad c_{jk}\in\mathds{C}$$ siendo la matriz $(c_{jk})$ unitaria. Definimos la función $V(\xi,\eta) = \sum_{k=1}^{N_{n,d}}V_j(\xi)\overline{V_j(\eta)}$
y $V(A\xi,A\eta) = V(\xi,\eta), \quad \forall A \in \orto$.
Dados $\xi,\eta \in \sphere$ podemos encontrar $A\in\orto$ tal que, $A\xi=e_d, A\eta = te_d + (1-t^2)^{\frac{1}{2}}e_{d-1}$ para $t=\xi.\eta$. Entonces $V(\xi,\eta) = V(e_d, te_d + (1-t^2)^{\frac{1}{2}}e_{d-1})$ es una función de $t=\xi\eta$. Llamaremos a esta funcion $P_d(t)$. Fijado $\xi$, el "maping" $\eta \to \overline{P_d(\xi.\eta)}$ es una función en $\mathds{V}$, del mismo modo, fijado $\zeta$ el mapping $\eta \to P_{n,d}(\zeta.\eta)$ es una función en $\spharm$.%fumada gorda%
\end{proof}
\begin{cor}Para $m\neq n$, $\mathds{Y}_m^d \perp \mathds{Y}_n^d$
\end{cor}
\begin{proof}Sean $Y_m \in \mathds{Y}_m^d$ e  $Y_n \in \mathds{Y}_n^d$ restricciones sobre la esfera de  $H_m \in \mathds{Y}_m(\mathds{R}^d)$ y $H_m \in \mathds{Y}_n(\mathds{R}^d)$ respectivamente. Como $\triangle H_m(x) = \triangle H_n(x) = 0$ tenemos que $$
	\int_{||x||< 1}(H_m\triangle H_n-H_n\triangle H_m)dx = 0
	$$
Aplicando la fórmula de Green$$
\int_{\sphere}(H_m \frac{\partial H_n}{\partial r} - H_n \frac{\partial H_m}{\partial r}) d\sphere = 0$$
Además, por ser $H_m$ un polinomio homogéneo de grado m $$
\left.\frac{\partial H_m(x)}{\partial r}\right|_{x=\xi} = m Y_m(\xi),\quad \xi\in\sphere$$
Análogamente,
$$
\left.\frac{\partial H_n(x)}{\partial r}\right|_{x=\xi} = m Y_n(\xi),\quad \xi\in\sphere 
$$
Por tanto, $$
\sphint (n-m)Y_m(\xi)Y_n(\xi)dS^{d-1}(\xi) = 0
$$
Finalmente, como $m\neq n$, $$
\sphint Y_m(\xi)Y_n(\xi)dS^{d-1}(\xi) = 0
$$
\end{proof}
\section{Generando Bases Ortonormales para Espacios de Esféricos Armónicos.}
\begin{prop}Si $Y_{j,d-1}\in\mathds{Y}_j^{d-1}$ entonces $P{n,d,j}(t)Y_{j,d-1}(\xi_{(d-1)})\in\spharm$ en coordenadas polares.
\end{prop}
\begin{defn}
Para $d\ge3$ y $m\le n$ definimos el operador $$
P_{n,m} : \mathds{Y}_m^{d-1} \to \spharm
$$
como
\end{defn}
\begin{thm}Para $d\ge 3$ y $n\ge 0$ se tiene que $$
	\spharm = \mathds{Y}^d_{n,0} \oplus ... \oplus \mathds{Y}^d_{n,n}
$$
\end{thm}
\begin{proof}
En primer lugar,veamos que los subespacios $\mathds{Y}^d_{n,i}$ son ortogonales 2 a 2. Sea $0\le k,m\le n$ con $k \neq m$. Para cualesquiera $\mathds{Y}_{k,d-1}\in\mathds{Y}_k^{d-1},\mathds{Y}_{m,d-1}\in\mathds{Y}_m^{d-1}$, %igualdades guapas
Luego, $\mathds{Y}_{n,k} \perp \mathds{Y}_{n,m}$ para  $k \neq m$.
\end{proof}

\appendix
\chapter{La Función Gamma}\label{aped.A}
\chapter{Resultados básicos de la esfera.}\label{aped.B}
Usaremos $dV^d$ para elemento diferencial de volumen y $dS^{d-1}$ para elemento diferencial de superficie de la esfera.  $\mathds{S^{d-1}}$

\begin{prop}Para $d \ge 3$ y $\xi \in \mathds{S}^{d-1}$,con $\xi_{(d)} = te_d+\sqrt{1-t^2}\xi_{(d-1)},  t\in[-1,1]$ , se tiene que
	$$
	dS^{d-1}(te_d+\sqrt{1-t^2}\xi_{(d-1)}) = (1-t^2)^{\frac{d-3}{2}}dt  dS^{d-2}(\xi_{(d-1)})
	$$
	Equivalentemente,
	$$
	dS^{d-1} = (1-t^2)^{\frac{d-3}{2}}dt dS^{d-2}
	$$
\end{prop}
\begin{example}Sea d=3 y $\xi$ un punto genérico de la esfera. Usando coordenadas esféricas $$
	\xi_{(3)}=\begin{pmatrix}
	cos\phi sen\theta\\
	sen\phi sen\theta\\
	cos\theta\\
	\end{pmatrix}
	0 \le \phi \le 2\pi , 0 \le \theta \le \pi
	$$
	Sea $t=cos\theta$ entonces
	$$
	\xi_{(2)} = \begin{pmatrix}
	cos\phi\\
	sen\phi\\
	0\\
	\end{pmatrix}
	$$
	Por tanto,$ \xi_{(3)} = te_3 + \sqrt{1-t^2} \xi_{(2)}$ y $dS^1 = d\phi , dS^2 = dtd\phi$
	
\end{example}
Podemos usar la anterior proposición para el cálculo del área de la superficie de la esfera.
\begin{prop}Se verifica que
	$$
	|\mathds{S}^{d-1}| = \int_{\mathds{S}^{d-1}} dS^{d-1} = \frac{2\pi^\frac{d}{2}}{\Gamma(\frac{d}{2})}
	$$
\end{prop} 

\begin{prop}
	Sea $A\in\mathds{R}^{dxd}$ ortogonal entonces
	$$ dS^{d-1}(A\xi) =  dS^{d-1}(\xi)$$
	$$ dV^{d}(A\xi) =  dV^{d}(\xi)$$
\end{prop}

Llamamos $C(S^{d-1})$ al espacio de funciones continuas sobre  $S^{d-1}$. Este espacio es un espacio de Banach con la norma $ ||f||_\infty = sup \{ |f(\xi) : \xi\in \mathds{S}^{d-1}\}$. Llamaremos $L^2(S^{d-1})$ al espacio de funciones con cuadrado integrable en $S^{d-1}$. Dicho espacio es un Hilbert con el producto escalar$$ (f,g) = \int_{S^{d-1}} f\overline{g} dS^{d-1}
$$
Consideramos el espacio $C(S^{d-1})$ con la norma inducida por el producto escalar de $L^2(S^{d-1})$. Este espacio no es completo. Además, el cierre de $C(S^{d-1})$ respecto a dicha norma es $L^2(S^{d-1})$. Es decir, dado una función $f\in L^2(S^{d-1})$ existe una sucesión $\{f_n\} \subset C(S^{d-1})$ tal que ${f_n}\to f$

\begin{prop}Sean $\Omega_\delta = \{x\in\mathds{R}^d : |x|\in[1-\delta,1+\delta]\}$ y $f^*(x)= f(\frac{x}{|x|}),x\in\Omega_\delta$ y $k\in\mathds{N}$.Entonces $f$ es k veces diferenciable en $S^{d-1}$ cuando $f^*$ lo es.  
\end{prop}
\begin{defn}Definimos $C^k(S^{d-1}), k\in\mathds{N}\cup0$ como el espacio de funciones k veces diferenciables en $S^{d-1}$
\end{defn}
\begin{prop}$C^k(S^{d-1})$ es un espacio de Banach con la norma 
	$$
	||f||_{C^k(S^{d-1})} = ||f^*||_{C^k(\Sigma_\delta)}
	$$
\end{prop}
\begin{rem}Usaremos $||f||_\infty = ||f||_{C(S^{d-1})}$
	
\end{rem}
\chapter{Polinomios de Legendre}\label{aped.C}


\section{Fórmulas de Representación}
\subsection{Fórmula de Rodrigues}
\begin{thm}
	$$P_{n,d}(t) = (-1)^n \frac{\Gamma(\frac{d-1}{2}) }{2^n\Gamma(n+\frac{d-1}{2})}(1-t^2)^{\frac{3-d}{2}}(\frac{d}{dt})^n (1-t^2)^{n+\frac{d-3}{2}}, \quad d\ge2
	$$
\end{thm}
\begin{rem}\label{cte_Rod}
	A la constante $R_{n,d} = \frac{\Gamma(\frac{d-1}{2})}{2^n\Gamma(n+\frac{d-1}{2})}$ se le llama constante de Rodrigues
\end{rem}
\begin{example}
	\begin{itemize}
		\item Si d = 2, $$P_{n,2}(t) = (-1)^n \frac{2^n n!} {(2n)!}(1-t^2)^{\frac{1}{2}}(\frac{d}{dt})^n (1-t^2)^{n-\frac{1}{2}}, \quad n\in \mathds{N}_0$$ Una forma reducida se obtiene usando el polinomio de Chebyshev obteniendo que $P_{n,2}(t) = cos(n arccos t), t\in[-1,1]$
		\item Si d=3, $$P_{n,3}(t) = \frac{1} {2^n n!}(\frac{d}{dt})^n (t^2-1)^{n}, \quad n\in \mathds{N}_0$$
	\end{itemize}
\end{example}
\subsection{Fórmulas de Representación Integral.}
\begin{thm}Sea $n\in\mathds{N}_0$ y $d\ge3$, $$ 
	P_{n,d}(t) = \frac{|\mathds{S}^{d-3}|}{|\mathds{S}^{d-2}|}\int_{-1}^{1}[t+i(1-t^2)^{\frac{1}{2}}s]^n(1-s^2)^{\frac{d-4}{2}} ds, \quad t\in[-1,1]
	$$
\end{thm}
\begin{rem}Como consecuencia de la fórmula anterior se tiene que $P_{n,d}(-t) = (-1)^n P_{n,d}(t), t\in[-1,1]$, es decir $P_{n,d}(t)$ tiene la misma paridad que $n$.
\end{rem}
Podemos obtener otra fórmula de representación integral, usando funciones trigonométricas mediante el cambio de variable $s = tanh u, u\in\mathds{R}$
\begin{thm}Sea $n\in\mathds{N}_0$ y $d\ge3$, $$
		P_{n,d}(t) = \frac{|\mathds{S}^{d-3}|}{|\mathds{S}^{d-2}|}\int_{-1}^{1}\frac{(1-s^2)^{\frac{d-4}{2}}}{[t\pm i(1-t^2)^{\frac{1}{2}}s]^{n+d-2}} ds, \quad t\in(0,1]
	$$
\end{thm}
\section{Propiedades}
\begin{prop}
Si $f\in C^n([-1,1])$ entonces 
$$
\int_{-1}^{1} f(t)P_{n,d}(t)(1-t^2)^{\frac{d-3}{2}} dt = R_{n,d}\int_{-1}^{1} f^{(n)}(t)(1-t^2)^{n+\frac{d-3}{2}} dt
$$
siendo $R_{n,d}$ la constante de Rodrigues \hyperref[]{(Nota \ref{cte_Rod})}
\end{prop}
\begin{prop}$P_{n,d}(t)$ tiene n raíces distintas en (-1,1)
\end{prop}
\begin{prop}Los polinomios de Legendre satisfacen la siguiente relación de recurrencia
	\begin{gather*}
		P_{n,d}(t) = \frac{2n+d-4}{n+d-3}t	P_{n-1,d}(t) - \frac{n-1}{n+d-3}P_{n-2,d}(t), \qquad n\ge 2, d\ge2 \\
		P_{0,d}(t) = 1 , 	P_{1,d}(t) = t 
	\end{gather*}
\end{prop}
\begin{prop}
	\begin{gather*}
	(1-t^2)P'_{n,d}(t) = n[P_{n-1,d}(t)-tP_{n,d}(t)], \quad n \ge 1,d \ge 2, t \in [-1,1]
	\end{gather*}
\end{prop}
\begin{prop}Para $d\ge 2$
$$\sum_{n=0}^{\infty} N_{n,d}r^nP_{n,d}(t) = \frac{1-r^2}{(1+r^2-2rt)^\frac{d}{2}},\quad |r| < 1, t\in[-1,1] 
$$
\end{prop}
\begin{prop}
	\begin{gather*}
	P_{n,d}(0) = \frac{|\mathds{S}^{d-3}|}{|\mathds{S}^{d-2}|}\int_{-1}^{1} i^n s^n(1-s^2)^{\frac{d-4}{2}}ds \\
	P_{n,d}(-1) = (-1)^n
	\end{gather*}
\end{prop}
\begin{prop}
	$$
	|P_{n,d}(t)| < \frac{\Gamma(\frac{d-1}{2})}{\sqrt{\pi}}\left[\frac{4}{n(1-t^2)}\right]^{\frac{d-2}{2}},\quad n\in\mathds{N}_0,d\ge2,t\in(-1,1)$$
\end{prop}

\chapter{Polinomios de Gegenbauer}\label{aped.D}
\begin{defn}Sean $v\ge 0,n\in\mathds{N}_0$ se define el polinomio de Gegenbauer de grado n e índice v, como:
	$$C_{n,v}(t) = \binom{n+2v-1}{n}\frac{\Gamma(v+\frac{1}{2})}{\sqrt{\pi}\Gamma(v)}\int_{-1}^{1}\left[t+i(1-t^2)^{1/2}s\right]^n (1-s^2)^{v-1} ds$$ 
\end{defn}
\begin{prop}(Identidad de Gegenbauer.)$$
	\sum_{n=0}^{\infty} C_{n,v}(t) = \frac{1}{(1+r^2-2rt)^v}, \qquad |r|<1, t\in[-1,1]$$
\end{prop} % Polinomios esfericos
\label{ch:gradiente}
\chapter[Cálculo del Gradiente]
		{Cálculo del Gradiente}
\section{El gradiente de los armónicos esféricos.}
Para el cálculo del gradiente usaremos una expresión de la base en términos de los polinomios de Gegenbauer. De la proposición \hyperref[]{\ref{geb_rel}}, se deduce que esta base es equivalente a la calculada anteriormente.
\\
\begin{thm}
Sean, $T_{n}(t),U_{n}(t)$ los polinomios de Chebyshev de $1^{er}$ y 2º clase respectivamente.Y definimos
\begin{gather*}
 g_{0,n}(x_1,x_2) = (x_1^2+x_2^2)T_n^2(x_2(x_1^2+x_2^2)^{-1/2})
\\
g_{1,n-1}(x_1,x_2) = x_1(x_1^2+x_2^2)^{\frac{n-1}{2}}U_{n-1}(x_2(x_1^2+x_2^2)^{-1/2})
\end{gather*}
entonces, si tomamos $\textbf{n}=(n_1,...,n_d)$ con $n_1 = \{0,1\}$ se define
\begin{gather*}
Y_\textbf{n} = g_{n_1,n_2}(x_1,x_2)\prod_{j=3}^{d}(x_1^2+...+x_j^2)^{n_j/2}C_{n_j,\lambda_j}(x_j(x_1^2+...+x_j^2)^{-1/2})
\end{gather*}
donde $\lambda_j =\lambda_j(n_1,...,n_{j-1}) = \sum_{i=1}^{j-1}n_i + \frac{j-2}{2}$. Entonces $\{Y_n,|\textbf{n}|=n\}$ es una base de $\mathds{Y}_{n}^{d}$.
\end{thm}
\medskip

Tomamos, $F_n^\lambda (x) = (x_1^2+...+x_j^2)^{n/2}C_{n,\lambda}(\frac{x_d}{\sqrt{(x_1^2+...+x_j^2)}})$.
Si $x=(x_1,...,x_d), n=(n_1,...,n_d)$ y $x' = (x_1,...,x_{d-1}), n=(n_1,..n_{d-1})$.
$Y_n(x) = Y_{n'}(x') F_{n_d}^{\lambda_d} (x)$ siendo $ Y_{n'}(x')$ un esférico armónico de dimensión $d-1$ y grado $n-n_d$.
\begin{prop}Para $i=1,...,d-1$
	\begin{gather*}
	\partial_i  F_{n}^\lambda(x) = -2\lambda x_i F_{n-2}^{\lambda+1}(x) \\ 
	\partial_d F_{n}^{\lambda}(x) = (n+2\lambda-1)  F_{n-1}^{\lambda}(x)
	\end{gather*}
\end{prop}
\begin{proof}
	Sean $r=\sqrt{x_1^2+...+x_d^2},i=1,2,...,d-1$. Usando que $$\frac{d}{dx}C_{n,\lambda}(x) = 2\lambda C_{n-1,\lambda+1}(x)$$ y los apartados $(I)$ y $(II)$ de la Proposición \ref{propGg} entonces
	\begin{gather*}
	\begin{aligned}
	\partial_i F_{n}^{\lambda} &= x_i r^{n-2} \left[ n C_{n,\lambda}(\frac{x_d}{r})-2\lambda\frac{x_d}{r}C_{n+1,\lambda+1}(\frac{x_d}{r})\right] \\&= -2\lambda x_ir^{n-2}C_{n-2,\lambda+1}(\frac{x_d}{r})
	\end{aligned}
	\end{gather*}
	Además,
	\begin{gather*}
		\begin{aligned}
		\partial_d F_{n}^{\lambda}(x) &= r^{n-1}\left[n\frac{x_d}{r}C_{n,\lambda}(\frac{x_d}{r})+2\lambda(1-\frac{x_d^2}{r^2})C_{n-1,\lambda+1}(\frac{x_d}{r})\right] \\&= (n+2\lambda-1)r^{n-1}C_{n-1,\lambda}(\frac{x_d}{r})		
		\end{aligned}
	\end{gather*}
	
	
\end{proof}
Ahora, tomamos la proyección del espacio de los polinomios homogéneos al espacio de los armónicos esféricos $proj_{n,\sphere}^d : \mathds{H}_n^d \to \spharm$ tal que para $P_n\in \mathds{H}_n^d$ verifica 
$$ proj_{n}^d P = \sum_{j=0}^{\lfloor \frac{n}{2} \rfloor}\frac{1}{4^j j! (-n+2-\frac{d}{2})} ||x||^{2j} \triangle^jP
$$
lo que implica que para $Y_\textbf{n}$ se tiene
$$
proj_{n,\sphere}^d (x_iY_{n}(x)) = x_i Y_n(x)-\frac{1}{2(n+(d-2)/2)} ||x||^2 \partial_i Y_n(x)$$
\begin{prop}Sea $n'=|n'|=n-n_d$ e $i=1,2,...,d-1$
	\begin{gather*}
	\begin{aligned}
	\partial_i Y_n(x) &= -2\lambda_d proj_{n'+1,\sphere}^{d-1}(xY_{n'}(x'))F_{n_d-2}^{\lambda_d+1}(x)\\   &+\frac{(n_d+2\lambda_d-1)(n_d+2\lambda_d-2)}{(2\lambda_d-1)(2\lambda_d-2)}\partial_iY_{n'}(x')F_{n_d}^{\lambda_d-1}(x) \\ \quad
	\partial_dY_n(x) &= (n_d+2\lambda_d-1)Y_{n'}(x')F_{n_d-1}^{\lambda_d}(x)
	\end{aligned}
	\end{gather*}
\end{prop}
\begin{proof} Usando los resultados anteriores y que $2\lambda_d-1 = 2n'+d-3$
	$$
	\begin{aligned}\partial_i Y_(x) &= \partial_i Y_{n'}(x')F_{n_d}^{\lambda_d}(x)-2\lambda_d x_i Y_{n'}(x')F_{n_d-2}^{\lambda_d+1}\\ &= -2\lambda_d proj_{n'+1,\sphere}^{d-1}(x_iY_{n'}(x'))F_{n_d-2}^{\lambda_d+1}(x)\\ &+ \partial_iY_{n'}(x')\left[F_{n_d}^{\lambda_d}(x)-\frac{2\lambda_d}{2\lambda_d-1}||x'||^2F_{n_d-2}^{\lambda_d+1}(x)\right]\end{aligned}$$
	Además, como $||x'||^2 = r^2-x_d^2$ y en virtud de los apartados $(I)$ y $(IV)$ de la proposición \ref{propGg}
	$$\begin{aligned}
	F_{n_d}^{\lambda_d}(x) -\frac{2\lambda_d}{2\lambda_d-1}||x'||^2F_{n_d-2}^{\lambda_d+1}(x) &=   r^n_d\left[C_{n_d}^{\lambda}(\frac{x_d}{r}) - -\frac{2\lambda_d}{2\lambda_d-1}(1-\frac{x_d}{r^2})C_{n_d-2}^{\lambda_d+1}(\frac{x_d}{r})\right] \\ &=   \frac{(n_d+2\lambda_d-1)(n_d+2\lambda_d-2)}{(2\lambda_d-1)(2\lambda_d - 2)}r^{n_d}C_{n_d}^{\lambda_d-1}(\frac{x_d}{r})
	\end{aligned}
	$$
	Sustituyendo esta igualdad en la obtenida anteriormente, se prueba el resultado.
\end{proof}
\begin{prop}Sea $n'=|\textbf{n}'|=n-n_d$ e $i=1,...,d-1$
	\begin{gather*}
	\begin{aligned}
		proj_{n+1,\sphere}^d(x_iY_n(x)) &= \frac{\lambda_d}{n_d+\lambda_d} proj_{n'+1,\sphere}^{d+1}(x_iY_{n'}(x'))F_{n_d}^{\lambda_d +1}(x)+\\& \frac{(n_d+1)(n_d+2)}{(2\lambda_d-1)(2\lambda_d-2)2(n_d+\lambda_d)}\partial_iY_{n'}(x')F_{n_d+2}^{\lambda-1}(x)\\
		proj_{n+1,\sphere}^d(x_dY_n(x)) &= \frac{n_d+1}{2(n_d+\lambda_d)}Y_{n'}(x')F_{n_d+1}^{\lambda_d}(x)
	\end{aligned}
	\end{gather*}
\end{prop}
\begin{proof}
	$$proj_{n+1,\sphere}^d(x_iY_n(x)) = x_iY_{n'} F_{n_d}^{\lambda_d}(x)-\frac{r^2}{2(n_d+\lambda_d)}\partial_i(Y_{n'}(x')F_{n_d}^{\lambda_d}(x))$$
	Usando la Proposición 2.3 tenemos que
	\begin{gather*}
	\begin{aligned}
	&proj_{n+1,\sphere}^d(x_iY_n(x))= \\ & proj_{n'+1,\sphere}^{d-1}(x_iY_{n'}(x'))\left[F_{n_d}^{\lambda_d}+\frac{\lambda_d}{n_d+\lambda_d}r^2F_{n_d-2,\lambda_d+1}(x)\right]\\&+\partial_iY_{n'}(x')\left[\frac{||x'||^2}{2\lambda_d-1}F_{n_d}^{\lambda_d}(x)-\frac{(n_d+2\lambda_d-1)(n_d+2\lambda_d)-2}{2(n_d+\lambda_d)(2\lambda_d-1)(2\lambda_d-2)}r^2F_{n_d}^{\lambda_d-1}(x)\right]
	\end{aligned}
	\end{gather*} 
	
	Ahora, aplicando el apartado $(IV)$ de la Proposición \ref{propGg}
	\begin{gather*}
	\begin{aligned}
	F_{n_d}^{\lambda_d}+\frac{\lambda_d}{n_d+\lambda_d}r^2F_{n_d-2,\lambda_d+1}(x) &= r^{n_d}\left[C_{n_d}^{\lambda_d}(\frac{x_d}{r})+\frac{\lambda_d}{n_d+\lambda_d}C_{n_d-2}^{\lambda_d+1}(\frac{x_d}{r})\right] \\&= \frac{\lambda_d}{n_d+\lambda_d}r^{n_d}C_{n_d}^{\lambda_d+1}(\frac{x_d}{r})
	\end{aligned}
	\end{gather*}
	Por otro lado, aplicando el apartado $(I)$ de la Proposición \ref{propGg}
	\begin{gather*}
	\begin{aligned}
	&\frac{||x'||^2}{2\lambda_d-1}F_{n_d}^{\lambda_d}(x)-\frac{(n_d+2\lambda_d-1)(n_d+2\lambda_d)-2}{2(n_d+\lambda_d)(2\lambda_d-1)(2\lambda_d-2)}r^2F_{n_d}^{\lambda_d-1}(x) \\&= \frac{r^{n_d+2}}{2\lambda_d-1}\left[(1-\frac{x_d^2}{r^2})C_{n_d}^{\lambda_d}(\frac{x_d}{r})-\frac{(n_d+2\lambda_d-1)(n_d+2\lambda_d-2)}{2(n_d+\lambda_d)(2\lambda_d-2)}C_{n_d}^{\lambda_d - 1}(\frac{x_d}{r})\right] \\ &= \frac{(n_d+1)(n_d+2)}{2(n_d+\lambda_d)(2\lambda_d-1)(2\lambda_d-2)}r^{n_d+2}C_{n_d+2}^{\lambda_d-1}(\frac{x_d}{r})
	\end{aligned}
	\end{gather*}
	Uniendo ambas igualdades se prueba la primera igualdad de la proposición.
	\\Finalmente,
	\begin{gather*}
	\begin{aligned}
	&proj_{n+1,\sphere}^d(x_dY_n(x)) = x_dY_{n'}(x')F_{n_d}^{\lambda_d}(x)-\frac{r^2}{2(n_d+\lambda_d)}\partial_d(Y_{n'}(x')F_{n_d}^{\lambda_d}(x)) \\&=Y_{n'}(x')\left[x_dF_{n_d}^{\lambda_d}(x)-\frac{r^2}{2(n_d+\lambda_d)}(n_d+2\lambda_d-1)F_{n_d-1}^{\lambda_d}(x)\right] \\&= \frac{n_d+1}{2(n_d+\lambda_d)}Y_{n'}(x')F_{n_d+1}^{\lambda_d}(x)
	\end{aligned}
	\end{gather*}
\end{proof}
\begin{thm}
	Sea $n=(n_1,n_2,...,n_d)\in\mathds{N}_0^d$ con $n_1=\{0,1\}$ y $|n|=n$. Entonces
	$\partial_i Y_n(x)$ es un esférico armónico de grado n-1 $$
	<\partial_i Y_n,Y_m>_{\sphere} \neq 0 \quad |m|=n-1$$
	para sólo $2^{d-2}$ índices, $m\in\mathds{N}_0^d$ con $m_1=\{0,1\}$
\end{thm}
\begin{proof}
La afirmación del teorema equivale a $\partial_i Y_n = \sum_{m} a_mY_m^{n-1}$ siendo $a_m$ una constante real. El resultado se prueba por inducción sobre la dimensión $d$ usando las proposiciones anteriores.
Para $d=2$, 
\begin{gather*}
\begin{aligned}
\partial_1 Y_n^{(1)}(x) &= nY_{n-1}^{(1)}(x) \qquad \partial_2 Y_n^{(1)}(x) = -nY_{n-1}^{(2)}(x)
\\ \partial_1 Y_n^{(2)}(x) &= nY_{n-1}^{(2)}(x) \qquad \partial_2 Y_n^{(2)}(x) = nY_{n-1}^{(1)}(x)
\end{aligned}
\end{gather*}
Supongamos cierto el resultado para dimensión $d-1$. Entonces $\partial_i Y_{n'}(x')$ puede escribirse como combinación lineal de a lo sumo $2^{d-3}$ esféricos $Y_m'^{n'-1}$. Como $Y_m^{n'-1}F_{n_d}^{\lambda_d-1} = Y_{m_1,...m_{d-1},n_d}^{n-1}$, el resultado se obtiene aplicando la Proposición 2.2.
\end{proof}
\subsection{Caso particular d=3}
El espacio de los esféricos armónicos de grado n en dimensión 3 tiene dimensión 2n+1. Tomando coordenadas esféricas,
\begin{gather*}
\begin{aligned}
x_1 &= r \sen \theta \sen \phi\\
x_2 &= r \sen \theta \sen \phi\\
x_3 &= r \cos \theta\\
\end{aligned}
\\
0\le \theta \le \pi,0\le\phi\le2\pi,r>0
\end{gather*}
una base ortogonal de $\spharm$ viene dada por
\begin{equation}
	\left\lbrace
	\begin{array}{ll}
	Y^n_ {k,1} = r^n(\sen \theta)^kC_{n-k,k+1/2}(\cos \theta) \cos(k\phi),\quad 0\le k\le n \\
	Y^n_{k,2}(x) = r^{n-k}(\sen\theta)^kC_{n-k,k+1/2}(\cos\theta)\sen(k\phi), \quad 1\le k\le n
	\end{array}
	\right.
\end{equation}

\begin{prop} Para $k=0,...,n$
	\begin{gather*} 
		\begin{aligned}
			\partial_1Y^{n}_{k,1}(x) &= -\frac{(n+k)(n+k-1)}{2(2k-1)}Y^{n-1}_{k-1,2}(x)-(k+\frac{1}{2})Y^{n-1}_ {k+1,2}(x) \\
		\partial_2Y^{n}_{k,1}(x) &= \frac{(n+k)(n+k-1)}{2(2k-1)}Y^{n-1}_{k-1,1}(x)-(k+\frac{1}{2})Y^{n-1}_ {k+1,1}(x) \\
		\partial_3 Y_{k,1}^{n}(x) &=(n+k)Y_{k,1}^{n-1}(x)
			\end{aligned}
	\end{gather*}
 Para $k=1,...,n$
	\begin{gather*}
	\begin{aligned}
	\partial_1Y^{n}_{k,2}(x) &= \frac{(n+k)(n+k-1)}{2(2k-1)}Y^{n-1}_{k-1,1}(x)+(k+\frac{1}{2})Y^{n-1}_ {k+1,1}(x)\\
	\partial_2Y^{n}_{k,2}(x) &= \frac{(n+k)(n+k-1)}{2(2k-1)}Y^{n-1}_{k-1,2}(x)-(k+\frac{1}{2})Y^{n-1}_ {k+1,2}(x)\\
	\partial_3 Y_{k,2}^{n}(x) &=(n+k)Y_{k,2}^{n-1}(x)
		\end{aligned}
	\end{gather*}
\end{prop}
\section{Puntos críticos del gradiente.}
Finalmente, queremos conocer el número de puntos que anulan el gradiente, para ello haremos uso de las siguientes igualdades trigonométricas
\begin{gather*}
\begin{aligned}
\cos(k\phi) &= \sen(k\phi + \frac{\pi}{2}) \\
\sen (k\phi) &= \cos(k\phi - \frac{\pi}{2}) \\
\cos(k+1)\phi &= \cos (k\phi)\cos\phi - \sen k\phi \sen\phi \\
\cos(k-1)\phi &= \cos (k\phi)\cos\phi + \sen k\phi \sen\phi \\
\sen(k+1)\phi &= \sen(k\phi)\cos\phi + \sen(\phi)\cos(k\phi)\\
\sen(k-1)\phi &= \sen(k\phi)\cos\phi - \sen(\phi)\cos(k\phi)\\
\end{aligned}
\end{gather*}
Llamaremos  $c_{n,k},d_k$ a las constantes $\frac{(n+k)(n+k-1)}{2(2k-1)},k+\frac{1}{2}$ respectivamente
Igualando a 0 las parciales calculadas anteriormente tenemos que para $k\ge0$
\begin{gather}
\partial_3 Y_{k,1}^n(x) = (n+k)Y_{k,1}^{n-1}(x) = (n+k)(\sen \theta)^k C_{n-k-1,k+1/2}(\cos \theta) \cos k\phi = 0
\end{gather}
implica que ha de verificarse una de las siguientes igualdades
\begin{gather}
\left\{
\begin{array}{ll}
\sen \theta= 0 \\
\cos k\phi = 0\\
C_{n-k-1,k+1/2}(\cos \theta) = 0
\end{array}
\right.
\end{gather}
Si $\sen \theta = 0$ tendremos que $\theta=0$ o $\theta=\pi$.
\medskip

Ahora, suponemos que $\cos k\phi = 0 $, entonces
\begin{gather}
\begin{aligned}
 \partial_1  Y_{k,1}^n(x) &= (\sen\theta)^{k-1}[-c_{n,k}C_{n-k,k-1/2}(\cos \theta)\sen(k\phi)\cos\phi \\&+ d_k \sen^2\theta C_{n-k-2,k+3/2}(\cos \theta)\sen(k\phi)\cos\phi] \\ &= (\sen\theta)^{k-1}\sen(k\phi)\cos\phi[-c_{n,k}C_{n-k,k-1/2}(\cos \theta) 
\\&+ d_k \sen^2\theta C_{n-k-2,k+3/2}(\cos \theta)]
\end{aligned}
\end{gather}
\begin{gather}
\begin{aligned}
\partial_2  Y_{k,1}^n(x) &= (\sen\theta)^{k-1}[-c_{n,k}C_{n-k,k-1/2}(\cos \theta)\sen(k\phi)\sen\phi \\&+ d_k \sen^2\theta C_{n-k-2,k+3/2}(\cos \theta)\sen(k\phi)\sen\phi] \\ &= (\sen\theta)^{k-1}\sen(k\phi)\sen\phi[-c_{n,k}C_{n-k,k-1/2}(\cos \theta) \\&+ d_k \sen^2\theta C_{n-k-2,k+3/2}(\cos \theta)]
\end{aligned}
\end{gather}
Ahora, igualando ambas expresiones a 0
\begin{gather}
\begin{aligned}
	\partial_1  Y_{k,1}^n(x) &= \partial_2  Y_{k,1}^n(x)  \\ &= (\sen\theta)^{k-1}\sen(k\phi)\cos\phi[c_{n,k}C_{n-k,k-1/2}(\cos \theta) + d_k \sen^2\theta C_{n-k-2,k+3/2}(cos \theta)]  \\ &= (\sen\theta)^{k-1}sen(k\phi)\sen\phi[c_{n,k}C_{n-k,k-1/2}(\cos \theta) + d_k \sen^2\theta C_{n-k-2,k+3/2}(\cos \theta)] \\ &= 0
\end{aligned}
\end{gather}
Como $cos (k	\phi) = 0 $ entonces $sen  (k\phi) \neq 0$. Además, $sen \phi$ y $cos\phi$ no se anulan simultáneamente, luego de la expresión anterior se verifica que
\begin{gather*}
c_{n,k}C_{n-k}^{k-1/2}(cos \theta) + d_k sen^2\theta C_{n-k-2}^{k+3/2}(cos \theta) = 0
\end{gather*}
Haciendo el cambio $sen^2 \theta = 1-cos^2\theta$, y tomando como variable $t=cos\theta$, se tiene que la expresión anterior es un polinomio de grado a lo sumo $n-k$. Por tanto, tiene a lo sumo $n-k$ raíces.

\medskip

Consideremos el polinomio:

\begin{gather*}
Q_{n,k}(t) =(n-k)(n+k-1)C_{n-k,k-\frac{1}{2}}(t)+(2k-1)(1-t^2)C_{n-k-2,k+\frac{3}{2}}(t)
\end{gather*}
De la ecuación (I) de la Proposición \ref{propGg} deducimos que:
\begin{gather*}
(2k-1)(2k+1)C_{n-k-2,k+\frac{3}{2}}(t) = \left(C_{n-k,k-\frac{1}{2}}\right)'' (t)
\end{gather*}

De la ecuación diferencial de los polinomios de Gegenbauer (Proposición \ref{Ggdif}) obtenemos:
\begin{gather*}
(1-t^2)\left(C_{n-k,k-\frac{1}{2}}\right)'' (t) = 2kt\left(C_{n-k,k-\frac{1}{2}}\right)'(t)-(n-k)(n-k-1)C_{n-k,k-\frac{1}{2}}(t)
\end{gather*}

Por tanto,
\begin{gather*}
\begin{aligned}
Q_{n,k}(t) &= \left[(n+k)(n+k-1)-(n-k)(n+k-1)\right] C_{n-k,k-\frac{1}{2}}(t) \\&+ 2kt\left(C_{n-k,k-\frac{1}{2}}\right)'(t) \\&= 2k\left[(n+k-1)C_{n-k,k-\frac{1}{2}}(t)+ t\left(C_{n-k,k-\frac{1}{2}}\right)'(t)\right]
\end{aligned}
\end{gather*}
y usando de (III) en la Proposición \ref{propGg} queda
\begin{gather*}
Q_{n-k}(t) = 2k\left(C_{n-k+1,k-\frac{1}{2}}\right)'(t)
\end{gather*}
que tiene exactamente $n-k$ raíces reales y distintas contenidas en el intervalo $[1,-1]$.

En resumen, hemos encontrado $2k(n-k)+2$ valores que anulan las 3 parciales.

\medskip
Razonando análogamente, se obtiene la misma conclusión para las parciales de los armónicos esféricos del segundo tipo.

\bigskip

A continuación, se muestran algunos ejemplos de los puntos críticos obtenidos para distintos valores de $n$ y $k$.
\begin{figure}[H]
	\centering
	\includegraphics[scale=0.5]{img/gradient_20_9.png}
	\caption{Puntos críticos del gradiente para n=20 y k=9.}
\end{figure}

\begin{figure}[H]
	\centering
	\includegraphics[scale=0.5]{img/gradient_25_5.png}
	\caption{Puntos críticos del gradiente para n=25 y k=20.}
\end{figure}

\begin{figure}[H]
	\centering
	\includegraphics[scale=0.5]{img/gradient_30_5.png}
	\caption{Puntos críticos del gradiente para n=30 y k=5.}
\end{figure}
 % Calculo del gradiente
\chapter[Integración Numérica]{Integración Numérica}
%5.1,5.3,5.4,5.6%
En este capítulo vamos a obtener una aproximación numérica de la integral $$ I(f) = \int_{\mathds{S}^2} f(\eta) dS^2(\eta) $$. 
\section{}
En primer lugar tomaremos la siguientes coordenadas esféricas
$$ \eta (cos\phi sin\theta, sin\phi sin\theta,cos \theta), \quad 0\le \phi \le 2\pi, 0\le \theta \le \pi$$
 
Ahora, 
$$
I(f) = \int_{0}^{2\pi} \int_{0}^{\pi} f(cos\phi sin\theta, sin\phi sin\theta,cos \theta)sin\theta d\theta d\phi. 
$$

Una vez hemos simplificado la expresión de la integral podemos aplicar métodos de integración numérica de una variable a cada una de las integrales.Comenzaremos integrando respecto a $phi$
\medskip
Como el integrando es periódico en $\pi$ con periodo $2\pi$, usaremos la formula del trapecio.
%meter formula del trapecio guapa aqui%
$$
I(g)=\int_{0}^{2\pi} g = \frac{2\pi}{m} \sum_{j=1}^{m} g(j\frac{2\pi}{m})
$$

\begin{lem} Para $m\ge 2,k\ge 0$
	\begin{gather}
	\int_{0}^{2\pi} cos(k\phi)d\phi = 
	\end{gather}
	\begin{gather}
	\frac{2\pi}{m}\sum_{j=0}^{m-1}cos(k\frac{2j\pi}{m}) = 
	\end{gather}
	\begin{gather}
	\int_{0}^{2\pi} sen(k\phi)d\phi =  \frac{2\pi}{m}\sum_{j=1}^{m-1}sen(k\frac{2j\pi}{m}) =0 
	\end{gather}
\end{lem}
\begin{proof}
	
\end{proof}

Finalmente estudiaremos la convergencia. Para ello introduciremos el espacio $H^q(2\pi)$ como aquel de las funciones de cuadrado integrable en $(0,2\pi)$ que verifican que 
$$
||f||_q = \sqrt{|a_0|^2+\sum_{}^{}|k||a_k|^2} < +\infty
$$
siendo $a_k$ los coeficientes de la serie de Fourier.
El espacio $H^q(2\pi)$ es un espacio de Hilbert con el producto escalar 
$$ (f,g)_q = a_0b_0 + \sum_{k=1}^{\infty}|k|^{2q} a_kb_k $$ siendo $a_k,b_k$ los coeficientes de la serie de Fourier para f y g respectivamente. 

\begin{thm}Sean $q > \frac{1}{2}, g\in H^q(2\pi)$, entonces
	$$
	| I(g) - I_m(g) | \le \frac{\sqrt{4\pi\zeta(2q)}}{m^q} ||g||_q, \qquad m\ge 1
	$$  
	siendo $\zeta$ la función zeta,
	$$\zeta(s) = \sum_{j=1}^{\inf} \frac{1}{j^s}$$
\end{thm}

Por otro lado, estudiamos el valor de la integral $\int_{0}^{2\pi} f(cos\phi sen\theta,sen\phi sin\theta,cos\theta)sen\theta d\theta$. Hacemos el cambio de variable $z= cos\theta$, la integral queda:
$$
\int_{-1}^{1} f(cos\phi\sqrt{1-z^2},sen \phi\sqrt{1-z^2},z) dz 
$$
Aplicamos la integración de Gauss-Legendre en $-1<z<1$.
$$
I_n(f) = h \sum_{j=0}^{2n-1} \sum_{k=1}^{n} w_k f(cos \phi_j\sqrt{1-z^2}, sen\phi_j\sqrt{1-z^2},z) = h \sum_{j=0}^{2n-1} \sum_{k=1}^{n} w_k f(cos \phi_jsen\theta_k, sen\phi_jsen\theta_k,cos\theta_k)
$$
\begin{thm}
	
\end{thm}
\begin{proof}
	
\end{proof}
\section{Métodos de Gauss de Orden Superior.}
En el caso de integración en una variable los métodos gaussianos se basan en pedir que la fórmula sea exacta para polinomios del mayor grado posible. Si tenemos n nodos es posible alcanzar esa exactitud para polinomios de grado $2n-1$. Este enfoque generaliza la integracion en varias variables. Sea $I(f)= \int f(\eta)dS^2(\eta) ~~ I_N(f) = \sum_{k=1}^{N} w_kf(\eta_k)$

${n_k},{w_k}$ se eligen para que los esféricos armónicos sean exactos para el grado mayor posible.

\begin{thm}
	Sea G un grupo de rotaciones...
\end{thm}
\begin{proof}
\end{proof}

\subsection{Eficiencia.}

\subsection{Método de los centroides}

\section{Integración scatered data}
Supongamos que tenemos N nodos, $P={n_1,...,n_N}$ y sus valores aproximados $f_i~f(n_i)$. Queremos aproximar la integral $I(f) =  \int_{\mathds{S}^2} f(n)dS^2(n)$.

\medskip
Tomamos $T_N={\triangle_1,...,\triangle_{M(N)}}$ la triangulación de $\mathds{S}^2$, donde los vértices de la triangulación son los nodos.

$$
I(f) = \sum_{k=1}^{M} \int_{\triangle_k} f(n)dS^2(n) =  \sum_{k=1}^{M} \frac{1}{3}[f(n_{k,1})+f(n_{k,2})+f(n_{k,3})] area(\triangle_k)
$$

\begin{thm}
\end{thm}
\begin{proof}
\end{proof}

f es lipschitziana en $\mathds{S}^2$ con constante $ c_f$
¿como elegir una triangulacion buena?
\section{Integración sobre el disco unidad.}
Finalmente, integraremos sobre el disco unidad $\mathds{D}={(x,y):x^2+y^2 \le 1}.$
La semiesfera superior es la imagen de 
$z=\sqrt{1-x^2-y^2} \qquad (x,y)\in \mathds{D}$
$\int_{D}f(x,y,\sqrt{1-x^2-y^2})\sqrt{1+(\frac{\partial z}{\partial x})^2+(\frac{\partial z}{\partial y})^2} dx dy = \int_D f(x,y,\sqrt{1-x^2-y^2})\frac{dx dy}{\sqrt{1-x^2-y^2}}$
Por tanto,
$$ \int_{\mathds{S}^2}f(\eta) dS^2(\eta) = \int_D \left[f(x,y,\sqrt{1-x^2-y^2})+f(x,y,-\sqrt{1-x^2-y^2})\right]\frac{dx dy}{\sqrt{1-x^2-y^2}} $$
Es decir, la integración sobre la esfera es equivalente a una integración con pesos sobre el disco unidad.


$$I(f)=\int_{D} f(x,y) dxdy = \int_{0}^{2\pi}\int_{0}^{1} rf(rcos\theta,rsen\theta)drd\theta $$
Para integrar sobre $\theta$ usamos la regla del trapecio y para hacerlo respecto de $r$ usamos la integración de Gauss-Legendre al integrando.

$$ I_n(f) = h\sum_{j=0}^{2n}\sum_{j=0}^{n} w_k r_k f(r_k cos\theta_j,r_ksen\theta_j)$$
\begin{thm}
\end{thm}
\begin{proof}
\end{proof} %integracion numerica
\appendix

\chapter{La Función Gamma}\label{aped.A}
\begin{defn} Dado $x\in\mathds{(R)^+}$ definimos la función gamma como
	$$
	\Gamma(x) := \int_{0}^{\infty} t^{x-1}e^{-t}dt		
	$$
\end{defn}
\begin{prop}Se verifican las siguientes formulas:
	$$
	\int_{0}^{\infty}  t^{x-1}e^{-at^b}dt = b^{-1}a^{-x/b}\Gamma(x/b)  , x,a,b \in \mathds{R}^+
	$$
	
	$$
	\int_{0}^{1} |ln t|^{x-1}dt = \Gamma(x),   x \in \mathds{R}^+
	$$
	
	$$
	\Gamma(x+1) = x \Gamma(x) ,		x\in \mathds{R}^+
	$$
	
	$$
	\Gamma^{(k)}(x) = \int_{0}^{\infty} (ln t)^k t^{x-1} e^{-t} dt,   k\in\mathds{N}_0,x\in\mathds{R}^+
	$$
\end{prop}
\begin{rem}
	$\Gamma(1)=1$ y de la tercera fórmula se deduce que $\Gamma(n+1)=n!, n\in\mathds{N}_0$. Es decir, la función $\Gamma$ extiende el operador factorial de los números naturales a los reales positivos.
\end{rem}
\begin{lem} 
	$$
	\Gamma(\frac{1}{2}) = 	\sqrt{\pi}
	$$
	$$
	\Gamma(n+\frac{1}{2})=\frac{(2n)!}{2^{2n}n!} \sqrt{\pi}
	$$
\end{lem}
\begin{defn}Sea $x\in\mathds{R}$ y $n\in\mathds{N}$,el símbolo de Pochhammer se define como
	$$
	(x)_0 = 1, (x)_n=x(x+1)(x+2)...(x+n-1)
	$$
\end{defn}
\begin{prop} Sea $x\in\mathds{R}^+$ entonces
	$$
	(x)_n = \frac{\Gamma(x+n)}{\Gamma(x)}
	$$
\end{prop}
\chapter{Resultados básicos de la esfera.}\label{aped.B}
Usaremos $dV^d$ para elemento diferencial de volumen y $dS^{d-1}$ para elemento diferencial de superficie de la esfera.  $\mathds{S^{d-1}}$

\begin{prop}Para $d \ge 3$ y $\xi \in \mathds{S}^{d-1}$,con $\xi_{(d)} = te_d+\sqrt{1-t^2}\xi_{(d-1)},  t\in[-1,1]$ , se tiene que
	$$
	dS^{d-1}(te_d+\sqrt{1-t^2}\xi_{(d-1)}) = (1-t^2)^{\frac{d-3}{2}}dt  dS^{d-2}(\xi_{(d-1)})
	$$
	Equivalentemente,
	$$
	dS^{d-1} = (1-t^2)^{\frac{d-3}{2}}dt dS^{d-2}
	$$
\end{prop}
\begin{example}Sea d=3 y $\xi$ un punto genérico de la esfera. Usando coordenadas esféricas $$
	\xi_{(3)}=\begin{pmatrix}
	cos\phi sen\theta\\
	sen\phi sen\theta\\
	cos\theta\\
	\end{pmatrix}
	0 \le \phi \le 2\pi , 0 \le \theta \le \pi
	$$
	Sea $t=cos\theta$ entonces
	$$
	\xi_{(2)} = \begin{pmatrix}
	cos\phi\\
	sen\phi\\
	0\\
	\end{pmatrix}
	$$
	Por tanto,$ \xi_{(3)} = te_3 + \sqrt{1-t^2} \xi_{(2)}$ y $dS^1 = d\phi , dS^2 = dtd\phi$
	
\end{example}
Podemos usar la anterior proposición para el cálculo del área de la superficie de la esfera.
\begin{prop}Se verifica que
	$$
	|\mathds{S}^{d-1}| = \int_{\mathds{S}^{d-1}} dS^{d-1} = \frac{2\pi^\frac{d}{2}}{\Gamma(\frac{d}{2})}
	$$
\end{prop} 

\begin{prop}
	Sea $A\in\mathds{R}^{dxd}$ ortogonal entonces
	$$ dS^{d-1}(A\xi) =  dS^{d-1}(\xi)$$
	$$ dV^{d}(A\xi) =  dV^{d}(\xi)$$
\end{prop}

Llamamos $C(S^{d-1})$ al espacio de funciones continuas sobre  $S^{d-1}$. Este espacio es un espacio de Banach con la norma $ ||f||_\infty = sup \{ |f(\xi) : \xi\in \mathds{S}^{d-1}\}$. Llamaremos $L^2(S^{d-1})$ al espacio de funciones con cuadrado integrable en $S^{d-1}$. Dicho espacio es un Hilbert con el producto escalar$$ (f,g) = \int_{S^{d-1}} f\overline{g} dS^{d-1}
$$
Consideramos el espacio $C(S^{d-1})$ con la norma inducida por el producto escalar de $L^2(S^{d-1})$. Este espacio no es completo. Además, el cierre de $C(S^{d-1})$ respecto a dicha norma es $L^2(S^{d-1})$. Es decir, dado una función $f\in L^2(S^{d-1})$ existe una sucesión $\{f_n\} \subset C(S^{d-1})$ tal que ${f_n}\to f$

\begin{prop}Sean $\Omega_\delta = \{x\in\mathds{R}^d : |x|\in[1-\delta,1+\delta]\}$ y $f^*(x)= f(\frac{x}{|x|}),x\in\Omega_\delta$ y $k\in\mathds{N}$.Entonces $f$ es k veces diferenciable en $S^{d-1}$ cuando $f^*$ lo es.  
\end{prop}
\begin{defn}Definimos $C^k(S^{d-1}), k\in\mathds{N}\cup0$ como el espacio de funciones k veces diferenciables en $S^{d-1}$
\end{defn}
\begin{prop}$C^k(S^{d-1})$ es un espacio de Banach con la norma 
	$$
	||f||_{C^k(S^{d-1})} = ||f^*||_{C^k(\Sigma_\delta)}
	$$
\end{prop}
\begin{rem}Usaremos $||f||_\infty = ||f||_{C(S^{d-1})}$
	
\end{rem}
\chapter{Polinomios de Legendre}\label{aped.C}
\section{Fórmulas de Representación}
\subsection{Fórmula de Rodrigues}
\begin{thm}
	$$P_{n,d}(t) = (-1)^n \frac{\Gamma(\frac{d-1}{2}) }{2^n\Gamma(n+\frac{d-1}{2})}(1-t^2)^{\frac{3-d}{2}}(\frac{d}{dt})^n (1-t^2)^{n+\frac{d-3}{2}}, \quad d\ge2
	$$
\end{thm}
\begin{rem}\label{cte_Rod}
	A la constante $R_{n,d} = \frac{\Gamma(\frac{d-1}{2})}{2^n\Gamma(n+\frac{d-1}{2})}$ se le llama constante de Rodrigues
\end{rem}
\begin{example}
	\begin{itemize}
		\item Si d = 2, $$P_{n,2}(t) = (-1)^n \frac{2^n n!} {(2n)!}(1-t^2)^{\frac{1}{2}}(\frac{d}{dt})^n (1-t^2)^{n-\frac{1}{2}}, \quad n\in \mathds{N}_0$$ Una forma reducida se obtiene usando el polinomio de Chebyshev obteniendo que $P_{n,2}(t) = cos(n \quad arccos t), t\in[-1,1]$
		\item Si d=3, $$P_{n,3}(t) = \frac{1} {2^n n!}(\frac{d}{dt})^n (t^2-1)^{n}, \quad n\in \mathds{N}_0$$
	\end{itemize}
\end{example}
\subsection{Fórmulas de Representación Integral.}
\begin{thm}Sea $n\in\mathds{N}_0$ y $d\ge3$, $$ 
	P_{n,d}(t) = \frac{|\mathds{S}^{d-3}|}{|\mathds{S}^{d-2}|}\int_{-1}^{1}[t+i(1-t^2)^{\frac{1}{2}}s]^n(1-s^2)^{\frac{d-4}{2}} ds, \quad t\in[-1,1]
	$$
\end{thm}
\begin{rem}Como consecuencia de la fórmula anterior se tiene que $P_{n,d}(-t) = (-1)^n P_{n,d}(t), t\in[-1,1]$, es decir $P_{n,d}(t)$ tiene la misma paridad que $n$.
\end{rem}
Podemos obtener otra fórmula de representación integral, usando funciones trigonométricas mediante el cambio de variable $s = tanh(u), u\in\mathds{R}$
\begin{thm}Sea $n\in\mathds{N}_0$ y $d\ge3$, $$
		P_{n,d}(t) = \frac{|\mathds{S}^{d-3}|}{|\mathds{S}^{d-2}|}\int_{-1}^{1}\frac{(1-s^2)^{\frac{d-4}{2}}}{[t\pm i(1-t^2)^{\frac{1}{2}}s]^{n+d-2}} ds, \quad t\in(0,1]
	$$
\end{thm}
\section{Propiedades}
\begin{prop}
Si $f\in C^n([-1,1])$ entonces 
$$
\int_{-1}^{1} f(t)P_{n,d}(t)(1-t^2)^{\frac{d-3}{2}} dt = R_{n,d}\int_{-1}^{1} f^{(n)}(t)(1-t^2)^{n+\frac{d-3}{2}} dt
$$
siendo $R_{n,d}$ la constante de Rodrigues \hyperref[]{(Nota \ref{cte_Rod})}
\end{prop}
\begin{prop}$P_{n,d}(t)$ tiene n raíces distintas en (-1,1)
\end{prop}
\begin{prop}Los polinomios de Legendre satisfacen la siguiente relación de recurrencia
	\begin{gather*}
		P_{n,d}(t) = \frac{2n+d-4}{n+d-3}t	P_{n-1,d}(t) - \frac{n-1}{n+d-3}P_{n-2,d}(t), \qquad n\ge 2, d\ge2 \\
		P_{0,d}(t) = 1 , 	P_{1,d}(t) = t 
	\end{gather*}
\end{prop}
\begin{prop}
	\begin{gather*}
	(1-t^2)P'_{n,d}(t) = n[P_{n-1,d}(t)-tP_{n,d}(t)], \quad n \ge 1,d \ge 2, t \in [-1,1]
	\end{gather*}
\end{prop}
\begin{prop}Para $d\ge 2$
$$\sum_{n=0}^{\infty} N_{n,d}r^nP_{n,d}(t) = \frac{1-r^2}{(1+r^2-2rt)^\frac{d}{2}},\quad |r| < 1, t\in[-1,1] 
$$
\end{prop}
\begin{prop}
	\begin{gather*}
	P_{n,d}(0) = \frac{|\mathds{S}^{d-3}|}{|\mathds{S}^{d-2}|}\int_{-1}^{1} i^n s^n(1-s^2)^{\frac{d-4}{2}}ds \\
	P_{n,d}(-1) = (-1)^n
	\end{gather*}
\end{prop}
\begin{prop}
	$$
	|P_{n,d}(t)| < \frac{\Gamma(\frac{d-1}{2})}{\sqrt{\pi}}\left[\frac{4}{n(1-t^2)}\right]^{\frac{d-2}{2}},\quad n\in\mathds{N}_0,d\ge2,t\in(-1,1)$$
\end{prop}

\chapter{Polinomios de Gegenbauer}\label{aped.D}

\begin{defn}Sean $v\ge 0,n\in\mathds{N}_0$ se define el polinomio de Gegenbauer de grado n e índice v, como:
	$$C_{n,v}(t) = \binom{n+2v-1}{n}\frac{\Gamma(v+\frac{1}{2})}{\sqrt{\pi}\Gamma(v)}\int_{-1}^{1}\left[t+i(1-t^2)^{1/2}s\right]^n (1-s^2)^{v-1} ds$$ 
\end{defn}

\begin{prop}\label{geb_rel}Se verifica la siguiente relación
	$$C_{n,\frac{d-2}{2}}(t) = \begin{pmatrix}
	n+d-3 \\
	b
	\end{pmatrix} P_{n,d}(t)
	$$
\end{prop}
\begin{prop}(Identidad de Gegenbauer.)$$
	\sum_{n=0}^{\infty} C_{n,v}(t) = \frac{1}{(1+r^2-2rt)^v}, \qquad |r|<1, t\in[-1,1]$$
\end{prop}
\chapter{Funciones de Legendre Asociadas}\label{aped.E}

Las funciones asociadas de Legendre nos permiten construir esféricos armónicos a partir de otros de menor dimensión.
\begin{defn}Sea $d\ge3$ y $n,j \in \mathds{N}_0$ se define la función asociada de Legendre de grado n y orden j en dimensión d, como
	$$
	P_{n,d,j}(t) = \frac{|\mathds{S}^{d-3}|}{|\mathds{S}^{d-2}|}i^{-j} \int_{-1}^{1}\left[t+i(1-t^2)^{1/2}s\right]^n P_{j,d-1}(s)(1-s^2)^{\frac{d-4}{2}}, \quad t\in[-1,1]
	$$
\end{defn}
\begin{rem}Si $j=0, P_{n,d,0}(t)=P_{n,d}(t)$
\end{rem}
%Las funciones asociadas de Legendre nos permiten generar "sistemas" de esféricos armónicos en la esfera.
\begin{prop}Sea $d\le3$ y $0\le j \le n$ 
	$$	P_{n,d,j}(t) = \frac{n!\Gamma(\frac{d-1}{2})}{2^j(n-j)!\Gamma(j+\frac{d-1}{2})}(1-t^2)^{1/2} P_{n-j,d+2j}(t), t\in[-1,1]$$
\end{prop}
El siguiente resultado nos proporciona una relación entre las funciones asociadas de Legendre y las derivadas de los polinomios de Legendre.
\begin{prop}Sea $d\le3$ y $0\le j \le n$ 
	$$	P_{n,d,j}(t) = \frac{(n+d-3)!}{(n+j+d-3)!}(1-t^2)^{1/2} P^{(j)}_{n,d}(t), t\in[-1,1]$$
\end{prop}
\begin{prop}
	\begin{gather*}
	\int_{-1}^{1} P_{m,d,j}(t)P_{n,d,j}(t)(1-t^2)^{\frac{d-3}{2}} dt = 0, \qquad m \neq n
	\end{gather*}
\end{prop}
\begin{prop} Las funciones $\tilde{P}_{n,d,j}$ definidas como
	\begin{gather*}
	\tilde{P}_{n,d,j}(t) = \frac{[(2n+d-2)(n-j)!(n+d+j-3)!]^{1/2}}{2^{\frac{d-2}{2}n!\Gamma(\frac{d-1}{2})}}P_{n,d,j}(t), \quad t\in[-1,1]
	\end{gather*}
	están normalizadas, es decir $\int_{-1}^{1} [	\tilde{P}_{n,d,j}]^2(1-t^2)^\frac{d-3}{2} dt = 1$
\end{prop}
\begin{rem}\label{note:fun_leg}Las funciones $	\tilde{P}_{n,d,j}$ pueden ser escritas en función de las derivadas de los polinomios de Legendre 
	\begin{gather*}
	\tilde{P}_{n,d,j}(t) =\frac{(n+d-3)!}{n!\Gamma(\frac{d-1}{2})} \frac{[(2n+d-2)(n-j)!]^{1/2}}{2^{d-2}(n+d+j-3)!}(1-t^2)^{j/2}P^{(j)}_{n,d}(t), \quad t\in[-1,1]
	\end{gather*}
\end{rem}

%\begin{thebibliography}{100}
\addcontentsline{toc}{chapter}{Bibliograf\'{\i}a}


\bibitem{At3} 
K.~E. Atkinson,
\newblock {\it An introduction to Numerical Analysis}, 2nd ed.,
\newblock Wiley, New York, 1989.

\bibitem{BF} 
L.~R. Burden, D.~J. Faires,
\newblock {\it An\'alisis Num\'erico}, S\'eptima Edici\'on,
\newblock Thomson Learning, M\'exico, 2003.


\bibitem{Ga13} 
W. Gautschi,
\newblock{\it Numerical analysis}, 2nd ed.,
\newblock Birkh\"auser--Science Springer, New York Dordrecht Heidelberg London 2012.

\bibitem{KCh}
D. Kincaid, W. Cheney,
\newblock {\it  An\'alisis Num\'erico. Las matem\'aticas del c\'alculo
cient\'{\i}fico},
\newblock Addison--Wesley Iberoamericana, Wilmington, 1994.

\bibitem{Ra} 
A. Ralston,
\newblock {\it Introducci\'on al An\'alisis Num\'erico},
\newblock Limusa--Wiley, M\'exico, 1970.

\bibitem{stewart} 
G. W. Stewart,
\newblock {\it Afternotes on Numerical Analysis},
\newblock SIAM, Philadelphia, 1996.

\bibitem{SB} 
J. Stoer, R. Burlirsch,
\newblock {\it Introduction to numerical analysis}, 3rd. ed.,
\newblock Springer, New York, 2002.

\end{thebibliography}




\end{document}

