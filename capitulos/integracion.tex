\chapter[Integración Numérica]{Integración Numérica}
%5.1,5.3,5.4,5.6%
En este capítulo vamos a obtener una aproximación numérica de la integral $$ I(f) = \int_{\mathds{S}^2} f(\eta) dS^2(\eta) $$. 
\section{}
En primer lugar tomaremos la siguientes coordenadas esféricas
$$ \eta (cos\phi sin\theta, sin\phi sin\theta,cos \theta), \quad 0\le \phi \le 2\pi, 0\le \theta \le \pi$$
 
Ahora, 
$$
I(f) = \int_{0}^{2\pi} \int_{0}^{\pi} f(cos\phi sin\theta, sin\phi sin\theta,cos \theta)sin\theta d\theta d\phi. 
$$

Una vez hemos simplificado la expresión de la integral podemos aplicar métodos de integración numérica de una variable a cada una de las integrales.Comenzaremos integrando respecto a $phi$
\medskip
\begin{rem}Regla del trapecio.
	$$\int_{a}^{b} f(x)dx ~ \frac{b-a}{b}\left[f(a)+2f(a+h)+2f(a+2h)+...+f(b)\right]
	$$
\end{rem}

Como el integrando es periódico en $\pi$ con periodo $2\pi$, usando la regla del trapecio tenemos que

$$
\widetilde{I}(g)\equiv \int_{0}^{2\pi} g(\phi)d\phi \approx \widetilde{I_m}(g) \equiv \frac{2\pi}{m} \sum_{j=1}^{m} g(j\frac{2\pi}{m})
$$

\begin{lem} Para $m\ge 2,k\ge 0$
	\begin{gather}
	\int_{0}^{2\pi} cos(k\phi)d\phi = \left\{\begin{array}{ll} 2\pi, \qquad k=0 
											\\ 0 ,  \qquad\  k>0
		\end{array} 
		\right.
	\end{gather}
	\begin{gather}
	\frac{2\pi}{m}\sum_{j=0}^{m-1}cos(k\frac{2j\pi}{m}) = \left\{\begin{array}{ll} 2\pi, \qquad k=0,m,2m,... 
	\\ 0 ,  \qquad en\ otro\ caso
	\end{array} 
	\right.
	\end{gather}
	\begin{gather}
	\int_{0}^{2\pi} sen(k\phi)d\phi =  \frac{2\pi}{m}\sum_{j=1}^{m-1}sen(k\frac{2j\pi}{m}) =0 
	\end{gather}
\end{lem}
\begin{proof}
	%falta demostracion
\end{proof}

Finalmente estudiaremos la convergencia. Para estudiar la convergencia de funciones periódicas, introduciremos el espacio $H^q(2\pi)$ como aquel de las funciones de cuadrado integrable en $(0,2\pi)$ que verifican que 
$$
||f||_q = \sqrt{|a_0|^2+\sum_{}^{}|k||a_k|^2} < +\infty
$$
siendo $a_k$ los coeficientes de la serie de Fourier.
El espacio $H^q(2\pi)$ es un espacio de Hilbert con el producto escalar 
$$ (f,g)_q = a_0b_0 + \sum_{k=1}^{\infty}|k|^{2q} a_kb_k $$ siendo $a_k,b_k$ los coeficientes de la serie de Fourier para f y g respectivamente. 

\begin{thm}Sean $q > \frac{1}{2}, g\in H^q(2\pi)$, entonces
	$$
	| I(g) - I_m(g) | \le \frac{\sqrt{4\pi\zeta(2q)}}{m^q} ||g||_q, \qquad m\ge 1
	$$  
	siendo $\zeta$ la función zeta,
	$$\zeta(s) = \sum_{j=1}^{\inf} \frac{1}{j^s}$$
\end{thm}
%¿demostracion?

Por otro lado, estudiamos el valor de la integral $\int_{0}^{2\pi} f(cos\phi sen\theta,sen\phi sin\theta,cos\theta)sen\theta d\theta$. Hacemos el cambio de variable $z= cos\theta$, la integral queda:
$$
\int_{-1}^{1} f(cos\phi\sqrt{1-z^2},sen \phi\sqrt{1-z^2},z) dz 
$$
\begin{rem}Integración de Gauss-Legendre
	$$
	\int_{-1}^{1}f(x)dx \approx \sum_{i=1}^{n} w_if(x_i)
	$$
\end{rem}
Aplicamos la integración de Gauss-Legendre en $-1<z<1$.
$$
I_n(f) = h \sum_{j=0}^{2n-1} \sum_{k=1}^{n} w_k f(cos \phi_j\sqrt{1-z^2}, sen\phi_j\sqrt{1-z^2},z) = h \sum_{j=0}^{2n-1} \sum_{k=1}^{n} w_k f(cos \phi_jsen\theta_k, sen\phi_jsen\theta_k,cos\theta_k)
$$
siendo ${z_i},{w_i}$ los nodos y los pesos de la fórmula de Gauss-Legendre respectivamente.
\begin{thm}Sea $f\in$ un polinomio esférico de grado menor o igual a $2n-1$. Entonces $I(f)=I_n(f)$. Además, para $f(x,y,z) = z^{2n}$,$I(f)\neq I_n(f)$
 	
\end{thm}
\begin{proof}
	Supongamos $f(x,y,z)= x^ry^sz^t, \quad r+s+t \le 2n-1$. Haciendo el cambio a coordenadas esféricas:
	$$
	I = \int_{\mathds{S}^2 x^ry^sz^t dS^2 = \int_{0}^{2\pi} \int_{0}^{\pi} cos^r\phi sen^{r+s+1}\theta sen^s \phi cos^t \theta d\phi d\theta
	$$
	y la correspondiente aproximación numérica
	$$
	I_m = \sum_{j=1}^{n} \sum_{k=1}^{n} h w_k cos^r \phi_j sen^s \phi_j sen^{r+s+1}\theta_k cos^t \theta_k
	$$
	Analizamos el error, $ E_n = I-I_n $
	%continua la magia%
\end{proof}
% error%

\section{Métodos de Gauss de Orden Superior.}
En el caso de integración en una variable los métodos gaussianos se basan en pedir que la fórmula sea exacta para polinomios del mayor grado posible. Si tenemos n nodos es posible alcanzar esa exactitud para polinomios de grado $2n-1$. Este enfoque generaliza la integracion en varias variables. Sea $I(f)= \int f(\eta)dS^2(\eta) ~~ I_N(f) = \sum_{k=1}^{N} w_kf(\eta_k)$

${n_k},{w_k}$ se eligen para que los esféricos armónicos sean exactos para el grado mayor posible.

\begin{thm}
	Sea G un grupo de rotaciones...
\end{thm}
\begin{proof}
\end{proof}

\subsection{Eficiencia.}

\subsection{Método de los centroides}

\section{Integración scatered data}
Supongamos que tenemos N nodos, $P={n_1,...,n_N}$ y sus valores aproximados $f_i~f(n_i)$. Queremos aproximar la integral $I(f) =  \int_{\mathds{S}^2} f(n)dS^2(n)$.

\medskip
Tomamos $T_N={\triangle_1,...,\triangle_{M(N)}}$ la triangulación de $\mathds{S}^2$, donde los vértices de la triangulación son los nodos.

$$
I(f) = \sum_{k=1}^{M} \int_{\triangle_k} f(n)dS^2(n) =  \sum_{k=1}^{M} \frac{1}{3}[f(n_{k,1})+f(n_{k,2})+f(n_{k,3})] area(\triangle_k)
$$
%error analisis%
%bla bla bla%
$|I(f)-I_n(f)|\le 4\pi c_f h$ siendo $h=max diam(\triangle), \quad \triangle\in T_N$
Luego,f es lipschitziana en $\mathds{S}^2$ con constante $ c_f$

Una pregunta que se nos plantea es: ¿como debemos elegir la triangulación?.
%triangulaciones%
\section{Integración sobre el disco unidad.}
Finalmente, integraremos sobre el disco unidad $\mathds{D}={(x,y):x^2+y^2 \le 1}.$
La semiesfera superior es la imagen de 
$z=\sqrt{1-x^2-y^2} \qquad (x,y)\in \mathds{D}$
$\int_{D}f(x,y,\sqrt{1-x^2-y^2})\sqrt{1+(\frac{\partial z}{\partial x})^2+(\frac{\partial z}{\partial y})^2} dx dy = \int_D f(x,y,\sqrt{1-x^2-y^2})\frac{dx dy}{\sqrt{1-x^2-y^2}}$
Por tanto,
$$ \int_{\mathds{S}^2}f(\eta) dS^2(\eta) = \int_D \left[f(x,y,\sqrt{1-x^2-y^2})+f(x,y,-\sqrt{1-x^2-y^2})\right]\frac{dx dy}{\sqrt{1-x^2-y^2}} $$
Es decir, la integración sobre la esfera es equivalente a una integración con pesos sobre el disco unidad.


$$I(f)=\int_{D} f(x,y) dxdy = \int_{0}^{2\pi}\int_{0}^{1} rf(rcos\theta,rsen\theta)drd\theta $$
Para integrar sobre $\theta$ usamos la regla del trapecio y para hacerlo respecto de $r$ usamos la integración de Gauss-Legendre al integrando.

$$ I_n(f) = h\sum_{j=0}^{2n}\sum_{j=0}^{n} w_k r_k f(r_k cos\theta_j,r_ksen\theta_j)$$
\begin{thm}Sea f(x,y) un polinomio de grado menor o igual a $2n$, entonces $I(f)=I_n(f)$. Además, la fórmula anterior tiene exactitud $2n$. 
\end{thm}
\begin{proof}
	Supongamos $f(x,y)=x^\alpha y^\beta$ con  $\alpha,\beta$ enteros positivos y tales que  $\alpha+\beta \le m$, entonces
	$$
	I(f)=\left(\int_{0}^{2\pi}(cos\theta)^{\alpha}(sen\theta)^{\beta}d\theta\right)\left(\int_{0}^{1}r^{\alpha+\beta +1}\right) \equiv J^{\alpha,\beta}K^{ \alpha+\beta+1 }
	$$
	$$
	I_n(f) = \left(h\sum_{j=0}^{2n}(cos\theta_j)^{\alpha}(sen\theta_j)^{\beta}\right) \left(\sum_{j=0}^{n}w_kr_k^{\alpha+\beta+1}\right)\equiv J_n^{\alpha,\beta}K_n^{ \alpha+\beta+1 }
	$$
	siendo
	$$
	k^t = \int_{0}^{1 } r^t \qquad k_n^t=\sum_{k=0}^{n} w_k{r_k}^t
	$$
	
	
	%cositas para la J%
	La fórmula de Gauss-Legendre para (n+1) puntos tiene exactitud 2n+1,luego $K_n^{ \alpha+\beta+1 }=K^{ \alpha+\beta+1 }, \quad \alpha+\beta \le 2n$.
	\medskip
	
	Por tanto, hemos probado que $I(f) = I_n(f) \quad \forall \alpha,\beta\ge 0, 0\le\alpha+\beta \le 2n $.
	Para comprobar que $I_n(f)$ tiene exactitud 2n basta considerar la función $f(x,y) = r(rcos\theta)^{2n+1}$. En este caso, $J^{2n+1,0}=0$ y $J_n^{2n+1,0},K_n^{2n+2}$ no se anulan.
\end{proof}