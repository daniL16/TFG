\chapter[Integración Numérica]{Integración Numérica}
En este capítulo vamos a obtener una aproximación numérica de la integral $$ I(f) = \int_{\mathds{S}^2} f(\eta) dS^2(\eta). $$. 
\section{Fórmulas de una variable.}
En primer lugar tomaremos la siguientes coordenadas esféricas
$$ \eta \mapsto (\cos\phi \sin\theta, \sin\phi \sin\theta,\cos \theta), \quad 0\le \phi \le 2\pi, 0\le \theta \le \pi$$
 
Ahora, 
$$
I(f) = \int_{0}^{2\pi} \int_{0}^{\pi} f(\cos\phi \sin\theta, \sin\phi \sin\theta,\cos \theta)\sin\theta d\theta d\phi. 
$$

Una vez hemos simplificado la expresión de la integral podemos aplicar métodos de integración numérica de una variable a cada una de las integrales. Comenzaremos integrando respecto a $\phi$.
\medskip
\begin{rem}Regla del trapecio compuesta.
	$$\int_{a}^{b} f(x)dx \approx \frac{b-a}{n}\left[\frac{f(a)+f(b)}{2}+\sum_{k=1}^{n-1}f(a+k\frac{b-a}{n})\right].
	$$
\end{rem}

Como el integrando es periódico en $\pi$ con periodo $2\pi$, usando la regla del trapecio tenemos que

$$
\widetilde{I}(g)\equiv \int_{0}^{2\pi} g(\phi)d\phi \approx \widetilde{I_m}(g) \equiv \frac{2\pi}{m} \sum_{j=1}^{m} g(j\frac{2\pi}{m})
$$

\begin{lem} Para $m\ge 2,k\ge 0$
	
	\begin{gather*}
	\begin{aligned}
	\int_{0}^{2\pi} \cos(k\phi)d\phi &= \left\{\begin{array}{ll} 2\pi, \qquad k=0 
											\\ 0 ,  \qquad\  k>0
		\end{array} 
		\right.
	\\
	\frac{2\pi}{m}\sum_{j=0}^{m-1}\cos(k\frac{2j\pi}{m}) &= \left\{\begin{array}{ll} 2\pi, \qquad k=0,m,2m,... 
	\\ 0 ,  \qquad en\ otro\ caso
	\end{array} 
	\right.
	\\
	\int_{0}^{2\pi} \sen(k\phi)d\phi &=  \frac{2\pi}{m}\sum_{j=1}^{m-1}\sen(k\frac{2j\pi}{m}) =0 
	\end{aligned}
	\end{gather*}
\end{lem}
\begin{proof}
Basta con sustituir las siguientes expresiones en la fórmula del trapecio
\begin{gather*}
\cos(iw) = \frac{e^{iw}+e^{-iw}}{2} \\
\sen(iw) = \frac{e^{iw}-e^{-iw}}{2i}
\end{gather*} 
\end{proof}

Finalmente estudiaremos la convergencia de $\widetilde{I}(g)$ a $I(g)$. Para estudiar la convergencia de funciones periódicas, introduciremos el espacio $H^q(2\pi)$ como aquel de las funciones de cuadrado integrable en $(0,2\pi)$ que verifican que 
$$
||f||_q = \sqrt{|a_0|^2+\sum_{}^{}|k||a_k|^2} < +\infty
$$
siendo $a_k$ los coeficientes de la serie de Fourier.
El espacio $H^q(2\pi)$ es un espacio de Hilbert con el producto escalar 
$$ (f,g)_q = a_0b_0 + \sum_{k=1}^{\infty}|k|^{2q} a_kb_k $$ siendo $a_k,b_k$ los coeficientes de la serie de Fourier para f y g respectivamente. 

\begin{thm}Sean $q > \frac{1}{2}, g\in H^q(2\pi)$, entonces
	$$
	| I(g) - I_m(g) | \le \frac{\sqrt{4\pi\zeta(2q)}}{m^q} ||g||_q, \qquad m\ge 1
	$$  
	siendo $\zeta$ la función zeta de Riemann,
	$$\zeta(s) = \sum_{j=1}^{\infty} \frac{1}{j^s}$$
\end{thm}
%¿demostracion?ref[13,p.316]

Por otro lado, estudiamos el valor de la integral $$\int_{0}^{2\pi} f(\cos\phi \sen\theta,\sen\phi \sin\theta,\cos\theta)\sen\theta d\theta$$. Para ello, hacemos el cambio de variable $z= \cos\theta$, la integral queda:
$$
\int_{-1}^{1} f(\cos\phi\sqrt{1-z^2},\sen \phi\sqrt{1-z^2},z) dz 
$$
\begin{rem}Integración de Gauss-Legendre
	$$
	\int_{-1}^{1}f(x)dx \approx \sum_{i=1}^{n} w_if(x_i)
	$$
\end{rem}
Aplicamos la integración de Gauss-Legendre en $-1<z<1$.
\begin{gather}\label{gaussian_quadrature}
\begin{aligned}
I_n(f) &= h \sum_{j=0}^{2n-1} \sum_{k=1}^{n} w_k f(\cos \phi_j\sqrt{1-z^2}, \sen\phi_j\sqrt{1-z^2},z) \\ &= h \sum_{j=0}^{2n-1} \sum_{k=1}^{n} w_k f(\cos \phi_j\sen\theta_k, \sen\phi_j\sen\theta_k,\cos\theta_k)
\end{aligned}
\end{gather}
siendo ${z_i},{w_i}$ los nodos y los pesos de la fórmula de Gauss-Legendre respectivamente.
\begin{thm}Sea $f$ un polinomio esférico de grado menor o igual a $2n-1$. Entonces $I(f)=I_n(f)$. Además, para $f(x,y,z) = z^{2n}$,$I(f)\neq I_n(f)$
 	
\end{thm}
\begin{proof}
	
	Supongamos $f(x,y,z)= x^ry^sz^t, \quad r+s+t \le 2n-1$. Haciendo el cambio a coordenadas esféricas:
	$$
	I = \int_{\mathds{S}^2} x^ry^sz^t dS^2 = \int_{0}^{2\pi} \int_{0}^{\pi} \cos^r\phi \sen^{r+s+1}\theta \sen^s \phi \cos^t \theta d\phi d\theta
	$$
	Sean 
	\begin{gather*}
	\begin{aligned}
	I &= \int_{\esfera} x^r y^s z^t dU = J^{r,s}K^{r,s,t} \\
	J^{r,s} &= \int_{0}^{2\pi} \cos^r\phi \sen ^s \phi d\phi \\
	K^{r,s,t} &= \int_{0}^{\pi} \sen^{r+s+1} \theta \cos^t \theta d\theta
	\end{aligned}
	\end{gather*}
	Para la correspondiente integral numérica tenemos
	y la correspondiente aproximación numérica
	\begin{gather}
	\begin{aligned}
		I_m &= \sum_{j=1}^{2n} \sum_{k=1}^{n} w_k x_{j,k}^r y_{j,k}^s z_{k}^t = J^{r,s}K^{r,s,t} \\
		J_n^{r,s} &= h \sum_{j=1}^{2n} \cos^r \phi_j \sen^s \phi_j \\
		K_n^{r,s,t} &= \sum_{k=1}^{n} w_k \sen^{r+s+1} \theta_k \cos^t \theta_k
	\end{aligned}
	\end{gather}

	Los valores $\{x_{j,k},y_{j,k},z_k\}$ representan las coordenadas cartesianas de los puntos obtenidos mediante las coordenadas esféricas $\{\phi_j\}$ y $\{\theta_j\}$.
	
	\medskip
	
	Ahora, analizaremos el error, $$ E_n = I-I_n = J^{r,s}K^{r,s,t}- J_n^{r,s}K_n^{r,s,t}$$
	
	Usaremos las siguientes propiedades trigonométicas, suponiendo que $r$ es impar y $s$ es par
	\begin{gather*}
	\begin{aligned}
	\cos^r(\pi+w) &= \cos^r(\pi-w) \\
	\sen^s(\pi+w) &= \sen^s(w) \\
	\sen^s (\frac{\pi}{2}+w) &= \sen^s (\frac{\pi}{2}-w) \\
	\cos^r (\frac{\pi}{2}+w) &= -\cos^r (\frac{\pi}{2}-w)
	\end{aligned}
	\end{gather*}

De estas igualdades se deduce que
\begin{gather*}
\begin{aligned}
J^{r,s} = J_n^{r,s} = 0
\end{aligned}
\end{gather*}
Razonando análogamente se obtiene la misma igualdad cuando $r$ es par y $s$ impar o cuando ambos son impares. En consecuencia $I-I_n = 0$.

\medskip

En \cite[p.170]{libro_esfarm} se demuestra que si  $f(x,y,z) = z^{2n}$ entonces $I(f)\neq I_n(f)$.
\end{proof} 
Finalmente, obtendremos una cota del error. Para ello haremos uso del Teorema 4.5 \cite[p.142]{libro_esfarm} y del corolario del Teorema 4.11 \cite[p.149]{libro_esfarm}.
\medskip

El error minimax de la aproximación de una función $f\in C(\esfera)$ por un polinomio esférico de grado menor o igual a m, se define como 
$$
E_{m,\infty}(f) = \min ||f-p||_{\infty}
$$
Sea $p_m^*$ un polinomio esférico de grado menor o igual a m cuyo error minimax se alcanza. La existencia de $p_m^*$ es consecuencia de los resultados citados anteriormente.
\medskip

De (\ref{gaussian_quadrature}) se deduce que $I(p_{2n-1}^*)=I_n(p_{2n-1}^*)$
Ahora, para $g\in C(\esfera)$
\begin{gather}
|I(g)| \le 4\pi||g||_\infty \\
|I_n(g)|\le 4\pi||g||_\infty
\end{gather}
y
\begin{gather}
I(f) - I_n(f) = I(f-p^*_{2n-1}) - I_n(f-p^*_{2n-1}) \\
|I(f) - I_n(f)| \le | I(f-p^*_{2n-1}) | + |  I_n(f-p^*_{2n-1}) | \le 8\pi||f-p^*_{2n-1}||_\infty
\end{gather}
\section{Métodos de Gauss de Orden Superior.}

En el caso de integración en una variable los métodos gaussianos se basan en pedir que la fórmula sea exacta para polinomios del mayor grado posible. Si tenemos n nodos es posible alcanzar esa exactitud para polinomios de grado $2n-1$. Este enfoque se generaliza para la integración en varias variables. \\
Sea $$
I(f)= \int f(\eta)dS^2(\eta) \approx I_N(f) = \sum_{k=1}^{N} w_kf(\eta_k)$$

Los nodos ${n_k}$, y los pesos ${w_k}$ se eligen de forma que la fórmula sea exacta para los armónicos esféricos de mayor grado posible.

\begin{thm}Sea $\mathcal{G}$ un grupo finito de rotaciones sobre la esfera. Y supongamos que el sistema anterior es invariante para todos los elementos de $\mathcal{G}$, es decir, $I_N(f_\gamma)=I_N(f), \quad \forall \gamma \in \mathcal{G}$. Entonces $$\{\eta_i:i=1,...,N\}=\{\gamma(\eta_i):i=1,...,N\}$$ Además ,para cada nodo $\eta_i$ sea $\mathcal{S}_i=\{\gamma(\eta_i):\gamma \in \mathcal{G}\}$.Entonces los pesos asociados a los nodos son iguales. Es condición necesaria y suficiente que f sea invariante para todo elemento de $\mathcal{G}$ para que  se verifique $I(f)=I_N(f)$. En tal caso, decimos que $I_N(f)$ tiene grado de precisión $d$, siendo $d$ la dimensión de f .

\end{thm}
\begin{proof}
Como la esfera es simétrica por rotaciones se verifica que $$I(f) = \int_{\esfera} f(\eta)dS^2(\eta) = \int_{\esfera} f_\gamma(\eta)dS^2(\eta) = I(f_\gamma)$$.
Sea $f^* = \frac{1}{|\mathcal{G}|} \sum_{\gamma \in |\mathcal{G}| } f\gamma$. $f^*$ es invariante por $\mathcal{G}$ e $I(f) = I(f^*)$. Por hipótesis del teorema, 
$$I(f)-I_N(f) = I(f^*) - I_N(f^*) $$
Por tanto, si queremos que $I(f)=I_N(f)$ para todo polinomio esférico $f$, basta con mostrar que el resultado es cierto para todo polinomio esférico que sea invariante por la acción de  $\mathcal{G}$.
\end{proof}
Razonando análogamente al apartado anterior se prueba que
$$
|I(f)-I_N(d)| \le 8\pi E_d(f)|
$$
y la velocidad con la que $I_N(f)$ converge a $I(f)$ depende de la regularidad de f.
\medskip

Los nodos y los pesos han de ser elegidos de forma que sean invariantes por $\mathcal{G}$. Además, el error $I(f)-I_N(d)$ debe ser 0 para polinomios del mayor grado posible, lo que requiere la resolución de un sistema de ecuaciones no lineal. Estas condiciones hacen que la elección de los nodos y los pesos sea una tarea complicada.
%algo mas?%
%¿meto esto tb?
%\subsection{Eficiencia.}
%\subsection{Método de los centroides}

\section{Integración puntos dispersos}
Supongamos que tenemos N nodos, $P=\{\eta_1,...,\eta_N\}$ y sus valores aproximados $f_i~\approx f(\eta_i)$. Queremos aproximar la integral $I(f) =  \int_{\mathds{S}^2} f(\eta)dS^2(\eta)$.

\medskip
Tomamos $T_N=\{\triangle_1,...,\triangle_{M(N)}\}$ la triangulación de $\mathds{S}^2$, donde los vértices de la triangulación son los nodos.

$$
I(f) = \sum_{k=1}^{M} \int_{\triangle_k} f(n)dS^2(n) \approx  \sum_{k=1}^{M} \frac{1}{3}[f(n_{k,1})+f(n_{k,2})+f(n_{k,3})] area(\triangle_k)
$$
Realizando un análisis del error cometido similar al realizado anteriormente, obtenemos que el método tiene grado de precisión 0 y 
$|I(f)-I_n(f)|\le 4\pi c_f h$ siendo $h=\max$ diam$(\triangle), \quad \triangle\in T_N$
Luego, f es lipschitziana en $\mathds{S}^2$ con constante $ c_f$

Finalmente, se nos plantean dos cuestiones: elegir una triangulación y un conjunto de nodos. Una de las opciones más frecuentes para la primera cuestión es la triangulación de Delaunay. En cuanto al conjunto de nodos, es conocido que se obtienen buenos resultados tomando un conjunto en el que los punto están bien distribuidos.
%meter referencias a otras triangulaciones y otros nodos$

\section{Integración sobre el disco unidad.}
Finalmente, integraremos sobre el disco unidad $\mathds{D}=\{(x,y):x^2+y^2 \le 1\}.$
La semiesfera superior es la imagen de 
$z=\sqrt{1-x^2-y^2} \qquad (x,y)\in \mathds{D}$

\begin{gather*}
\begin{aligned}
&\int_{D}f(x,y,\sqrt{1-x^2-y^2})\sqrt{1+(\frac{\partial z}{\partial x})^2+(\frac{\partial z}{\partial y})^2} dx dy \\ 
&= \int_D f(x,y,\sqrt{1-x^2-y^2})\frac{dx dy}{\sqrt{1-x^2-y^2}}
\end{aligned}
\end{gather*}
	
Por tanto,
\begin{gather*}
\begin{aligned}
&\int_{\mathds{S}^2}f(\eta) dS^2(\eta)  =\\ &\int_D \left[f(x,y,\sqrt{1-x^2-y^2})+f(x,y,-\sqrt{1-x^2-y^2})\right]\frac{dx dy}{\sqrt{1-x^2-y^2}}
\end{aligned}
\end{gather*}  
Es decir, la integración sobre la esfera es equivalente a una integración con pesos sobre el disco unidad.


$$I(f)=\int_{D} f(x,y) dxdy = \int_{0}^{2\pi}\int_{0}^{1} rf(r\cos\theta,r\sen\theta)drd\theta $$
Para integrar sobre $\theta$ usamos la regla del trapecio y para hacerlo respecto de $r$ usamos la integración de Gauss-Legendre al integrando.

$$ I_n(f) = h\sum_{j=0}^{2n}\sum_{j=0}^{n} w_k r_k f(r_k \cos\theta_j,r_k\sen\theta_j)$$
\begin{thm}Sea f(x,y) un polinomio de grado menor o igual a $2n$, entonces $I(f)=I_n(f)$. Además, la fórmula anterior tiene exactitud $2n$. 
\end{thm}
\begin{proof}
	Supongamos $f(x,y)=x^\alpha y^\beta$ con  $\alpha,\beta$ enteros positivos y tales que  $\alpha+\beta \le n$, entonces
	$$
	I(f)=\left(\int_{0}^{2\pi}(\cos\theta)^{\alpha}(\sen\theta)^{\beta}d\theta\right)\left(\int_{0}^{1}r^{\alpha+\beta +1}\right) \equiv J^{\alpha,\beta}K^{ \alpha+\beta+1 }
	$$
	$$
	I_n(f) = \left(h\sum_{j=0}^{2n}(\cos\theta_j)^{\alpha}(\sen\theta_j)^{\beta}\right) \left(\sum_{j=0}^{n}w_kr_k^{\alpha+\beta+1}\right)\equiv J_n^{\alpha,\beta}K_n^{ \alpha+\beta+1 }
	$$
	siendo
	$$
	K^t = \int_{0}^{1 } r^t \qquad K_n^t=\sum_{k=0}^{n} w_k{r_k}^t
	$$
	
	Por otro lado, si $\beta$ es impar las integrales $J^{\alpha,\beta},J_n^{\alpha,\beta}=0$ (visto en la demostración del Teorema 3.5). Si $\beta$ es par podemos transformar el integrando $(\cos \theta)^\alpha(\sen\theta)^\beta$ en un polinomio de potencias de $\cos \theta$ con grado $\alpha+\beta$. Usando el Lema 3.2 se tiene que $J_n^{\alpha,\beta} = J^{\alpha,\beta}, \alpha+\beta\le 2n$. 
	\medskip
	La fórmula de Gauss-Legendre para (n+1) puntos tiene exactitud 2n+1,luego $K_n^{ \alpha+\beta+1 }=K^{ \alpha+\beta+1 }, \quad \alpha+\beta \le 2n$.
	\medskip
	
	Por tanto, hemos probado que $I(f) = I_n(f) \quad \forall \alpha,\beta\ge 0, 0\le\alpha+\beta \le 2n $.
	Para comprobar que $I_n(f)$ tiene exactitud 2n basta considerar la función $f(x,y) = r(r\cos\theta)^{2n+1}$. En este caso, $J^{2n+1,0}=0$ y $J_n^{2n+1,0},K_n^{2n+2}$ no se anulan.
\end{proof}