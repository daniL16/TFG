\chapter[Cálculo del Gradiente]
		{Cálculo del Gradiente}
\section{Cálculo Teórico.}
Para el cálculo del gradiente usaremos una expresión de la base en términos de los polinomios de Gegenbauer. De \ref{geb_rel}, se deduce que esta base es equivalente a la calculada anteriormente.
Sean, $T_{n}(t),U_{n}(t)$ los polinomios de Chevyschev de $1^{er}$ y 2º orden respectivamente.	Y definimos
\begin{gather*}
 g_{0,n}(x_1,x_2) = (x_1^2+x_2^2)T_n^2(x_2(x_1^2+x_2^2)^{-1/2})
\\
g_{1,n-1}(x_1,x_2) = x_1(x_1^2+x_2^2)^{\frac{n-1}{2}}U_{n-1}(x_2(x_1^2+x_2^2)^{-1/2})
\end{gather*}
entonces, si tomamos $n=(n_1,...,n_d)$ con $n_1 = \{0,1\}$ se define
\begin{gather*}
Y_n = g_{n_1,n_2}(x_1,x_2)\prod_{j=3}^{d}(x_1^2+...+x_j^2)^{n_j/2}C_{n_j,\lambda_j}(x_j(x_1^2+...+x_j^2)^{-1/2})
\end{gather*}
donde $\lambda_j =\lambda_j(n_1,...,n_{j-1}) = \sum_{i=1}^{j-1}n_i + \frac{j-2}{2}$


Tomamos, $F_n^\lambda (x) = (x_1^2+...+x_j^2)^{n/2}C_{n,\lambda}(\frac{x_d}{\sqrt{(x_1^2+...+x_j^2)}})$.
Si $x=(x_1,...,x_d), n=(n_1,...,n_d)$ y $x' = (x_1,...,x_{d-1}), n=(n_1,..n_{d-1})$.
$Y_n(x) = Y_{n'}(x') F_{n_d}^{\lambda_d} (x)$ siendo $ Y_{n'}(x')$ un esférico armónico de dimensión $d-1$ y grado $n-n_d$.
\begin{prop}Para $i=1,...,d-1$
	\begin{gather*}
	\partial_i  F_{n}^\lambda(x) = -2\lambda x_i F_{n-2}^{\lambda+1}(x) \\ 
	\partial_d F_{n}^{\lambda}(x) = (n+2\lambda-1)  F_{n-1}^{\lambda}(x)
	\end{gather*}
\end{prop}
\begin{proof}
	Sea $r=\sqrt{x_1^2+...+x_d^2}$. Usando que $\frac{d}{dx}C_{n,\lambda}(x) = 2\lambda C_{n-1,\lambda+1}(x)$ entonces para $i=1,2,...,d-1$ se tiene que$$
	\partial_i F_{n}^{\lambda} = x_i r^{n-2} \left[ n C_{n,\lambda}(\frac{x_d}{r})-2\lambda\frac{x_d}{r}C_{n+1,\lambda+1}(\frac{x_d}{r})\right] = -2\lambda x_ir^{n-2}C_{n-2,\lambda+1}(\frac{x_d}{r})
	$$
	Además, $$
	\partial_d F_{n}^{\lambda}(x) = r^{n-1}\left[n\frac{x_d}{r}C_{n,\lambda}(\frac{x_d}{r})+2\lambda(1-\frac{x_d^2}{r^2})C_{n-1,\lambda+1}(\frac{x_d}{r})\right] = (n+2\lambda-1)r^{n-1}C_{n-1,\lambda}(\frac{x_d}{r})
	$$
\end{proof}

\begin{prop}Sea $n'=|n'|=n-n_d$ e $i=1,2,...,d-1$
	\begin{gather*}
	\partial_i Y_n(x) = -2\lambda_d proj_{n'+1,\sphere}^{d-1}(xY_{n'}(x'))F_{n_d-2}^{\lambda_d+1}(x)+\frac{(n_d+2\lambda_d-1)(n_d+2\lambda_d-2)}{(2\lambda_d-1)(2\lambda_d-2)}\partial_iY_{n'}(x')F_{n_d}^{\lambda_d-1}(x) \\
	\partial_dY_n(x) = (n_d+2\lambda_d-1)Y_{n'}(x')F_{n_d-1}^{\lambda_d}(x)
	\end{gather*}
\end{prop}
\begin{prop}
	De la proposición anterior se deduce que
	$$ \partial_i Y_n(x) = \partial_i Y_{n'}(x')F_{n_d}^{\lambda_d}(x) - 2 \lambda_dx_iY_{n'}(x')F_ {n_d-2}^{\lambda_d+1}(x) = -2\lambda_dproj_{n'+1,\sphere}^{d-1}(x_i Y_{n'}(x'))F_{n_d-2}^{\lambda_d+1}(x) + \partial_iY_{n'}(x')\left[F_{n_d}^{\lambda_d}(x)-\frac{2\lambda_d}{2\lambda_d-1}||x'||^2F_{n_d-2}^{\lambda_d+1}(x)\right]$$
	donde hemos usado que $2n'+d-3=2\lambda_d-1$. Como $||x'||^2=r^2-x_d^2$
	$$
	F_{n_d}^{\lambda_d}(x)-\frac{2\lambda_d}{2\lambda_d-1}||x'||^2F_{n_d-2}^{\lambda_d+1}(x) = r^{n_d}\left[C_{n_d}^{\lambda_d}(\frac{x_d}{r})-\frac{2\lambda_d}{2\lambda_d-1}(1-\frac{x_d^2}{r^2})C_{n_d-2}^{\lambda_d+1}(\frac{x_d}{r})\right] = \frac{(n_d+2\lambda_d-1)(n_d+2\lambda_d-2)}{(2\lambda_d-1)(2\lambda_d-2)}r^{n_d}C_{n_d}^{\lambda_d-1}(\frac{x_d}{r})
	$$
\end{prop}
\begin{thm}
	Sea $n=(n_1,n_2,...,n_d)\in\mathds{N}_0^d$ con $n_1=\{0,1\}$ y $|n|=n$. Entonces
	(1) $\partial_i Y_n(x)$ es un esférico armónico de grado n-1 $$
	<\partial_i Y_n,Y_m>_{\sphere} \neq 0 \quad |m|=n-1$$
	para al menos $2^{d-2}$ $m\in\mathds{N}_0^d$ con $m_1=\{0,1\}$
\end{thm}
\begin{proof}
	
\end{proof}
\subsection{Caso particular d=3}
El espacio de los esféricos armónicos de grado n en dimensión 3 tiene dimensión 2n+1. Tomando coordenadas esféricas,
$$
x_1 = r sen \theta sen \phi
x_2 = r sen \theta sen \phi
x_3 = r cos \theta
0\le \theta \le \pi,0\le\phi\le2\pi,r>0
$$
una base ortogonal de $\spharm$ es
$$
Y^n_ {k,1} = r^n(sen \theta)^kC_{n,k}^{k+1/2}(cos \theta) cos(k\phi), 0\le k\le n \\
Y^n_{k,2}(x) = r^{n-k}(sen\theta)^kC^{k+1/2}_{n-k}(cos\theta)sen(k\phi), 1\le k\le n
$$ 
\begin{prop} Para $k=0,...,n$
	\begin{gather*} 
		\begin{aligned}
			\partial_1Y^{n}_{k,1}(x) &= -\frac{(n+k)(n+k-1)}{2(2k-1)}Y^{n-1}_{k-1,2}(x)-(k+\frac{1}{2})Y^{n-1}_ {k+1,2}(x) \\
		\partial_2Y^{n}_{k,1}(x) &= \frac{(n+k)(n+k-1)}{2(2k-1)}Y^{n-1}_{k-1,1}(x)-(k+\frac{1}{2})Y^{n-1}_ {k+1,1}(x) \\
		\partial_3 Y_{k,1}^{n}(x) &=(n+k)Y_{k,1}^{n-1}(x)
			\end{aligned}
	\end{gather*}
 Para $k=1,...,n$
	\begin{gather*}
	\begin{aligned}
	\partial_1Y^{n}_{k,2}(x) &= \frac{(n+k)(n+k-1)}{2(2k-1)}Y^{n-1}_{k-1,1}(x)+(k+\frac{1}{2})Y^{n-1}_ {k+1,1}(x)\\
	\partial_2Y^{n}_{k,2}(x) &= \frac{(n+k)(n+k-1)}{2(2k-1)}Y^{n-1}_{k-1,2}(x)-(k+\frac{1}{2})Y^{n-1}_ {k+1,2}(x)\\
	\partial_3 Y_{k,2}^{n}(x) &=(n+k)Y_{k,2}^{n-1}(x)
		\end{aligned}
	\end{gather*}

	
\end{prop}