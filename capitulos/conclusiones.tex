\chapter{Conclusiones.}
Los resultados obtenidos en esta última sección no han sido tan satisfactorios como a priori esperaba, ya que la aplicación de algoritmos alternativos no ha mejorado los resultados. Esto pone en relieve el porqué la mayoría de los participantes se decanta por el uso de xgboost, ya que al ser una librería de código abierto tiene un gran soporte que permite que el algoritmo esté muy optimizado. Sin embargo, los resultados obtenidos por RUSBoost parecen prometedores.

\medskip 

Por otro lado, hemos constatado la importancia de la fase de preprocesamiento ya que hemos conseguido una gran mejora sin importar la capacidad de procesamiento de nuestras máquinas. Esto nos lleva a concluir que en situaciones donde la importancia de procesamiento no fuera tan relevante, habríamos obtenido una mejor clasificación final. 

\medskip

Tras la finalización de la competición se ha obtenido la posición 1327 de 3951 participantes. Personalmente me ha parecido muy interesante afrontar este problema y poder comparar mis soluciones con las propuestas por gente experimentada en este ámbito. Haber participado en esta competición me ha servido para ganar experiencia, que me será de gran ayuda en el futuro. El proyecto, con los scripts usados son accesibles públicamente en \href{https://github.com/daniL16/kaggle\_	talkingdata}{https://github.com/daniL16/kaggle\_talkingdata}.